
\newglossaryentry{elf}{
	name=ELF,
	description={Executable and Linkable Format; a file format used to store
	executable programs. An ELF file describes the address at which each
	section of program code or data should be loaded.}
}

\newglossaryentry{cr0}{
	name=\texttt{cr0},
	description={Control Register 0. It holds bits which determine how the
	processor operates. It e.g. determines whether paging and protected mode
	are enabled.}
}

\newglossaryentry{cr3}{
	name=\texttt{cr3},
	description={Control Register 3. It points to the current page table.}
}

\newglossaryentry{esp}{
	name=\texttt{esp},
	description={Extended Stack Pointer register.}
}

\newglossaryentry{eip}{
	name=\texttt{eip},
	description={Extended Instruction Pointer; a register which holds the
	address of the next instruction to be executed.}
}

\newglossaryentry{eflags}{
	name=\texttt{eflags},
	description={A register which holds various flags that mostly reflect the
	properties of the most recently executed instruction. For example the
	signed flag is set during a subtraction whose result is below zero.}
}

\newglossaryentry{cs}{
	name=\texttt{cs},
	description={Code Segment selector register; it holds an index into the
	\gls{gdt}. Its lower two bits determine the \gls{cpl}.}
}

\newglossaryentry{cpl}{
	name=CPL,
	description={Current Privilege Level; the current ring in which the
	processor is executing. CPL=3 means ring, 3, i.e., user-mode, while CPL=0
	means ring 0, i.e. kernel-mode.}
}

\newglossaryentry{gdt}{
	name=GDT,
	description={Global Descriptor Table; a table which holds descriptors,
	each of which describes a segment of memory and its permissions.}
}

\newglossaryentry{pte}{
	name=PTE,
	description={Page Table Entry; a 32-bit integer which stores a physical
	address to which a virtual address maps, as well as some status bits.}
}
\newglossaryentry{mmu}{
	name=MMU,
	description={Memory Management Unit.}
}
\newglossaryentry{tlb}{
	name=TLB,
	description={Translation Lookaside Buffer.}
}

\newglossaryentry{pde}{
	name=PDE,
	description={Page Directory Entry; a 32-bit integer which stores a
	physical address of a second-level node in the page table, as well as some
	status bits.}
}

\newglossaryentry{idt}{
	name=IDT,
	description={Interrupt Descriptor Table; a table which describes the
	address of the handler that should be run when a given interrupt or
	exception is triggered.}
}

\newglossaryentry{idtr}{
	name=IDTR,
	description={A register which holds the physical address of the \gls{idt}.}
}

\newglossaryentry{tr}{
	name=TR,
	description={Task Register; a register which is used as an index into the
	\gls{gdt} to find the \gls{tss}.}
}

\newglossaryentry{tss}{
	name=TSS,
	description={Task State Segment; a data structure which, among other
	things, determines the \gls{esp}-value used during a context switch
	triggered by an exception or interrupt.}
}

\newglossaryentry{pic}{
	name=PIC,
	description={Programmable Interrupt Controller; a hardware device which
	orders interrupts before delivering them to the processor}
}

\newglossaryentry{apic}{
	name=APIC,
	description={Advanced \gls{pic}.}
}

\newglossaryentry{lapic}{
	name=LAPIC,
	description={Local \gls{apic}. The LAPIC is the processor-local component
	of the \gls{apic}. Modern systems have one LAPIC per processor.}
}

\newglossaryentry{mmio}{
	name=MMIO,
	description={Memory-Mapped I/O. If you communicate with a device using
	MMIO, it means that the registers of the device are mapped into memory at
	some address, and so communication happens by reading from or writing to
	memory.}
}

\newglossaryentry{mpconfig}{
	name=mpconfig,
	description={mpconfig is a method for finding information about multiple
	processors described in Intel's Multi-processor specification.}
}

\newglossaryentry{bp}{
	name=BP,
	description={Bootstrap Processor; the first, and initially only, processor
	that runs when a system boots.}
}

\newglossaryentry{ap}{
	name=AP,
	description={Application Processor; any processor which is not a \gls{bp}.
	The APs only run once they are started by the \gls{bp}.}
}

\newglossaryentry{ipi}{
	name=IPI,
	description={Inter-Processor Interrupt. An interrupt sent by one processor
	to another using the \gls{lapic}.}
}

\newglossaryentry{pio}{
	name=PIO,
	description={Programmed Input/Output. A way to read and write data from/to
	disk (or another device) using instructions such as \texttt{inb} and
	\texttt{outb}. Often an alternative to \gls{mmio}.}
}

\newglossaryentry{ipc}{
	name=IPC,
	description={Inter-Process Communication. A mechanism which lets processes
	communicate.}
}

\newglossaryentry{pci}{
	name=PCI,
	description={Peripheral Component Interconnect. A type of bus to which
	devices, such as the network card, can be connected.}
}

\newglossaryentry{lfb}{
	name=LFB,
	description={Linear Frame Buffer, a buffer that repesents the pixels which
	are drawn to the screen.}
}

\makeglossaries
