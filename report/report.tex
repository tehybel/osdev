\documentclass{report}
%\documentclass[10pt,letterpaper]{scrartcl}

\usepackage[dottedtoc,manychapters,floatperchapter]{classicthesis}
\usepackage[utf8]{inputenc}
\usepackage{graphicx}
\usepackage{amsfonts}
\usepackage{amsmath}
\usepackage{framed}
\usepackage{hyperref}
\usepackage{fancyvrb}
\usepackage{color}
\usepackage{bm}
\usepackage{listings}
\usepackage{soul,xcolor}
\usepackage[htt]{hyphenat} % allows breaks inside texttt
\usepackage[acronym, toc]{glossaries}
\setstcolor{red}


% this makes sure that each section gets a fresh page.
% \usepackage{titlesec}
% \newcommand{\sectionbreak}{\clearpage}


\usepackage{xcolor}

% TODO: find a nicer way to indicate that text is a link.
\definecolor{light-grey}{gray}{0.30}

% this makes links prettier
\hypersetup{
    colorlinks,
	linktoc=page,
    linkcolor={light-grey}, % color of internal links (sections, pages, etc.)
    citecolor={blue!30!black},
    urlcolor={blue!30!black} % color of URL links (mail, web)
}

\lstset{language=C}
\lstset{basicstyle=\ttfamily\footnotesize}

\newglossaryentry{elf}{
	name=ELF,
	description={Executable and Linkable Format; a file format used to store
	executable programs. An ELF file describes the address at which each
	section of program code or data should be loaded.}
}

\newglossaryentry{cr0}{
	name=\texttt{cr0},
	description={Control Register 0. It holds bits which determine how the
	processor operates. It e.g. determines whether paging and protected mode
	are enabled.}
}

\newglossaryentry{cr3}{
	name=\texttt{cr3},
	description={Control Register 3. It points to the current page table.}
}

\newglossaryentry{esp}{
	name=\texttt{esp},
	description={Extended Stack Pointer register.}
}

\newglossaryentry{eip}{
	name=\texttt{eip},
	description={Extended Instruction Pointer; a register which holds the
	address of the next instruction to be executed.}
}

\newglossaryentry{eflags}{
	name=\texttt{eflags},
	description={A register which holds various flags that mostly reflect the
	properties of the most recently executed instruction. For example the
	signed flag is set during a subtraction whose result is below zero.}
}

\newglossaryentry{cs}{
	name=\texttt{cs},
	description={Code Segment selector register; it holds an index into the
	\gls{gdt}. Its lower two bits determine the \gls{cpl}.}
}

\newglossaryentry{cpl}{
	name=CPL,
	description={Current Privilege Level; the current ring in which the
	processor is executing. CPL=3 means ring, 3, i.e., user-mode, while CPL=0
	means ring 0, i.e. kernel-mode.}
}

\newglossaryentry{gdt}{
	name=GDT,
	description={Global Descriptor Table; a table which holds descriptors,
	each of which describes a segment of memory and its permissions.}
}

\newglossaryentry{pte}{
	name=PTE,
	description={Page Table Entry; a 32-bit integer which stores a physical
	address to which a virtual address maps, as well as some status bits.}
}
\newglossaryentry{mmu}{
	name=MMU,
	description={Memory Management Unit.}
}
\newglossaryentry{tlb}{
	name=TLB,
	description={Translation Lookaside Buffer.}
}

\newglossaryentry{pde}{
	name=PDE,
	description={Page Directory Entry; a 32-bit integer which stores a
	physical address of a second-level node in the page table, as well as some
	status bits.}
}

\newglossaryentry{idt}{
	name=IDT,
	description={Interrupt Descriptor Table; a table which describes the
	address of the handler that should be run when a given interrupt or
	exception is triggered.}
}

\newglossaryentry{idtr}{
	name=IDTR,
	description={A register which holds the physical address of the \gls{idt}.}
}

\newglossaryentry{tr}{
	name=TR,
	description={Task Register; a register which is used as an index into the
	\gls{gdt} to find the \gls{tss}.}
}

\newglossaryentry{tss}{
	name=TSS,
	description={Task State Segment; a data structure which, among other
	things, determines the \gls{esp}-value used during a context switch
	triggered by an exception or interrupt.}
}

\newglossaryentry{pic}{
	name=PIC,
	description={Programmable Interrupt Controller; a hardware device which
	orders interrupts before delivering them to the processor}
}

\newglossaryentry{apic}{
	name=APIC,
	description={Advanced \gls{pic}.}
}

\newglossaryentry{lapic}{
	name=LAPIC,
	description={Local \gls{apic}. The LAPIC is the processor-local component
	of the \gls{apic}. Modern systems have one LAPIC per processor.}
}

\newglossaryentry{mmio}{
	name=MMIO,
	description={Memory-Mapped I/O. If you communicate with a device using
	MMIO, it means that the registers of the device are mapped into memory at
	some address, and so communication happens by reading from or writing to
	memory.}
}

\newglossaryentry{mpconfig}{
	name=mpconfig,
	description={mpconfig is a method for finding information about multiple
	processors described in Intel's Multi-processor specification.}
}

\newglossaryentry{bp}{
	name=BP,
	description={Bootstrap Processor; the first, and initially only, processor
	that runs when a system boots.}
}

\newglossaryentry{ap}{
	name=AP,
	description={Application Processor; any processor which is not a \gls{bp}.
	The APs only run once they are started by the \gls{bp}.}
}

\newglossaryentry{ipi}{
	name=IPI,
	description={Inter-Processor Interrupt. An interrupt sent by one processor
	to another using the \gls{lapic}.}
}

\newglossaryentry{pio}{
	name=PIO,
	description={Programmed Input/Output. A way to read and write data from/to
	disk (or another device) using instructions such as \texttt{inb} and
	\texttt{outb}. Often an alternative to \gls{mmio}.}
}

\newglossaryentry{ipc}{
	name=IPC,
	description={Inter-Process Communication. A mechanism which lets processes
	communicate.}
}

\newglossaryentry{pci}{
	name=PCI,
	description={Peripheral Component Interconnect. A type of bus to which
	devices, such as the network card, can be connected.}
}

\newglossaryentry{lfb}{
	name=LFB,
	description={Linear Frame Buffer, a buffer that repesents the pixels which
	are drawn to the screen.}
}

\makeglossaries



%%%%%%%%%%%%%%%%%%%%%%%%%%%%%%%%%%%%%%%%%%%%%%%%%%%%%%%%%%%%%%%%%%%%%%%%%%%%%
%%%%%%%%%%%%%%%%%%%%%%%%%%%%%%%%%%%%%%%%%%%%%%%%%%%%%%%%%%%%%%%%%%%%%%%%%%%%%
%%%%%%%%%%%%%%%%%%%%%%%%%%%%%%%%%%%%%%%%%%%%%%%%%%%%%%%%%%%%%%%%%%%%%%%%%%%%%

\begin{document}

\pagenumbering{roman}
% TODO. the numbering is off.


\begin{titlepage} % Suppresses headers and footers on the title page

\centering % Centre everything on the title page

\rule{\textwidth}{1pt} % Thick horizontal rule
\vspace{2pt}\vspace{-\baselineskip} % Whitespace between rules
\rule{\textwidth}{0.4pt} % Thin horizontal rule
\vspace{0.1\textheight} % Whitespace between the top rules and title

{\Huge \bfseries
% TODO improve this title.
Operating System \\ Implementation
}

\vspace{0.010\textheight} % Whitespace between the title and short horizontal rule
\rule{0.3\textwidth}{0.4pt} % Short horizontal rule under the title
\vspace{0.01\textheight} % Whitespace between the thin horizontal rule and the author name

{\Large \textsc{Thomas Hybel}} \\[0.5\baselineskip]
{\Large \textsc{Aarhus University}} \\[0.5\baselineskip]
{\Large \textsc{November 2017}} 

\vfill % Whitespace between the author name and publisher



%------------------------------------------------
%	Bottom rules
%------------------------------------------------

\rule{\textwidth}{0.4pt} % Thin horizontal rule
\vspace{2pt}\vspace{-\baselineskip} % Whitespace between rules
\rule{\textwidth}{1pt} % Thick horizontal rule

\end{titlepage}
\newpage


%TODO: have your email in the title page


\begin{abstract} 
\noindent 
TODO: write an abstract here.
\end{abstract}
\newpage


\tableofcontents

\newpage
\pagenumbering{arabic}



%%%%%%%%%%%%%%%%%%%%%%%%%%%%%%%%%%%%%%%%%%%%%%%%%%%%%%%%%%%%%%%%%%%%%%%%%%%%%%

\chapter{Introduction}
This report documents our journey through writing an operating system from
scratch. Our primary goal was to learn about operating systems internals and
design, and to investigate the difficulty of the task. Neither usability nor
efficiency were significant goals throughout the process, except when
their consideration facilitated learning.

% TODO: add citation https://pdos.csail.mit.edu/6.828/2016/schedule.html
To get started on kernel development we made extensive use of material from
the MIT 2016 Operating System Engineering course, which is accessible online.
The MIT course is divided into six major "labs", with each lab encapsulating a
set of kernel functionality. For example, lab 4 involves functionality needed
for multitasking and concurrency, while lab 5 comprises the implementation of
a custom file system. 

The kernel we wrote is for the 32-bit x86 architecture. The kernel is written
in C, with some parts having to be hand-written in assembly. 
We have used GCC as our compiler and Git for version
control. Additionally, it was convenient to develop the kernel on emulated
hardware rather than a real machine, since this eased debugging and increased
development speed. It also made it trivial to change the amount of system RAM,
and to add new processors or an extra hard drive. We have used the full-system
emulator QEMU for this task.

Each MIT lab has an associated web page which describes what needs to be done
and provides links to resources like manuals and specifications. Since we opted
to follow the MIT course, this also meant that some design decisions were made
for us. One example of this is us following an exokernel philosophy, which
meant that we have pushed large parts of the kernel functionality into user
space.

During kernel development, occasionally some functionality was important,
but not technically interesting to implement. In these cases the MIT lab often
provided the code for us, with enough code missing that we had to understand
the concept to complete it, while still saving significant amounts of time. An
example of this is the code for communicating with the outside world over a
serial connection, which is used throughout the labs.


In addition to the six labs of the MIT course, we have decided upon two labs
on our own. In lab 7 we designed and implemented a graphical user interface
for the operating system. In lab 8 we made the operating system run on real
hardware instead of QEMU. Since we were given zero guidance nor code, these
labs were significantly more difficult and time-consuming than previous ones.

Each of the following sections corresponds directly to one lab. They describe
the development process, from first booting into a minimal kernel, until the end
when we run a full graphical operating system on a real machine.


% TODO get rid of the glossary, or at least minimize it. If we keep it, add a
% reference to it here in the intr.

% TODO: read each of the sections' introductions and see whether they make
% sense when read alone in sequence


% TODO rewrite "the goal of this lab was", it sounds bad and repetitive

% TODO: clarify OS vs kernel in the whole report.

%%%%% LAB 1 %%%%%


\chapter{Booting}
\label{sec:lab1}

In this lab we wrote initialization code and a boot loader for our kernel. 
The initialization code switches the processor to 32-bit protected mode. The
boot loader loads the rest of the kernel into memory and jumps to it.

\section{The boot process}
To understand the purpose of a boot loader, it helps to have an overview of
the startup process of an x86 machine.
When an x86 machine starts, its BIOS code runs. The BIOS initializes some of
the hardware, and then it loads the first sector (512 bytes) from the boot
medium into a hard-coded address which it jumps to.

This first sector will typically contain a small program known as the boot
loader. Its purpose is to load the main kernel from disk and transfer
execution to it.

Before the boot loader loads the kernel, the kernel should first run some
initialization code which sets up a more comfortable environment for the boot
loader and kernel to work in.


\section{Initialization code}
When the BIOS jumps into our code, the processor is running in 16-bit real
mode. However the code produced by a modern compiler expects to run in 32-bit
protected mode. The initialization code should therefore be written in
assembly and should switch processor modes.

The most important difference between real and protected mode is that real
mode does not support the use of a page table to implement virtual memory.

We switch to protected mode by setting a bit in the control register
\texttt{cr0}. Enabling virtual memory similarly works by setting another bit
in \texttt{cr0}. We do not immediately enable virtual memory, though, since we
do not yet have the infrastructure to set up a proper page table. This happens
in section \ref{sec:mem}. Until then, all addressing is physical.

To switch to 32-bit mode, we must update the \texttt{cs} (code segment)
register. The \texttt{cs} register is an offset into a table called the Global
Descriptor Table (\gls{gdt}) which is an array of descriptors. Each descriptor
describes a segment of memory; a bit in the descriptor determines whether the
segment contains 16-bit or 32-bit code. 

We use the \texttt{ljmp} instruction to update the \gls{cs} register and use a
32-bit segment descriptor. Next, the boot loader is executed; since the
processor is in 32-bit protected mode, the rest of the kernel code can be
written in C rather than assembly.

%The file \texttt{boot/boot.S} contains the code that performs the described
%tasks.



\section{The boot loader}
It is the task of the boot loader to load the main kernel and transfer
execution to it. Since the boot loader resides on the first sector of the hard
drive, it must load the kernel using no more than 512 bytes of code, minus the
bytes used by the initialization code.

Our kernel is compiled into an Execute and Linkable Format (ELF) file. The ELF
file specifies the physical addresses into which the code and data of the
kernel must be loaded. The boot loader must thus parse the ELF file to load
the kernel.

Once the boot loader has loaded the kernel, it determines the entry point
address from the ELF file and jumps there.


\section{Minimal kernel code}
At this point we were able to run kernel code. However we did not yet have any
functionality; unlike user-mode programs, the kernel does not have access to a
C standard library, unless we write one ourselves.

As a start, we wanted the ability to input and output text.
Our kernel uses the \texttt{inb} and \texttt{outb} instructions to
communicate with the outside world via a serial connection. We used this to
print a "hello, world" message and confirm that the kernel runs.

Using instructions such as \texttt{inb} and \texttt{outb} to communicate with
an input/output device is quite common. The method is known as Programmed
Input/Output (\gls{pio}); it will be used often in the following sections.


%%%%% LAB 2 %%%%%

\chapter{Memory management}
\label{sec:mem}
The goal of this lab was to enable virtual memory after setting up a page
table. To set up the page table, we first needed to implement a subsystem
for allocating and freeing pages of physical memory.


\section{Physical page management}
% sources: https://en.wikipedia.org/wiki/Nonvolatile_BIOS_memory
% http://wiki.osdev.org/CMOS
A given system has a limited amount of physical memory, depending on how much
RAM the machine has. This memory is split up into a number of pages. On x86 a
page is 4096 bytes, or \texttt{0x1000} in hexadecimal. Thus pages always aligned on
\texttt{0x1000}-byte boundaries. The kernel determines the amount of RAM by using
\gls{pio} to query a memory area called the CMOS. 

The physical page management subsystem will keep a reference count for each
page. If the count is zero, the page is free and can be allocated. This
metadata is stored in an array of \texttt{PageInfo} structs. Each entry in
this array directly corresponds to one physical page of memory, such that the
first \texttt{PageInfo} struct holds metadata about the first page of physical
memory, and so on.

The \texttt{PageInfo} of free pages are additionally stored in a linked list,
such that the kernel can return a free page in constant time.

We wrote the following functions to manage physical pages:
\begin{itemize}
\item \texttt{page\_alloc} is used to allocate a page of physical memory
\item \texttt{page\_free} is used to put a page on the free list
\item \texttt{page\_decref} and \texttt{page\_incref} are used to manage
reference counts of pages
\end{itemize}
These functions provide critical infrastructure needed by other kernel
features.



\section{Page table theory}
It is necessary to introduce some theory before we can explain how our kernel
initializes its page table.

The x86 page table is a two-level table whose main purpose is to let the
processor translate a virtual address to a physical address. The page table
can be thought of as a 1024-ary tree with two levels. 

A pointer to the first level of the page table can be found in the
\texttt{cr3} register. The first level is called the Page Directory. It
contains 1024 Page Directory Entries (\gls{pde}s). Each \gls{pde} points to a
second-level node, which contains 1024 Page Table Entries (\gls{pte}s). 
A \gls{pte} specifies a page of physical memory and its permissions, including
whether it is writable and whether it is accessible to user-mode code.

To translate from a virtual to a physical address, it is necessary to walk the
page table. Say that the process wishes to access a virtual address $v$. It
first looks in the \gls{cr3} register to find the Page Directory. It uses the
10 higher-order bits of $v$ to specify a \gls{pde}. It uses the next 10 bits
of $v$ to specify a \gls{pte}. The \gls{pte} contains the address of a
physical page. The last 12 bits of $v$ are used as an offset into this page,
and the address translation is complete. The processor also validates the
permissions of the page before the access, and generates a page fault if these
are inappropriate.

This is a costly process, and in practice the job is done by specialized
hardware called a Memory Management Unit (MMU). Additionally, recent
translations are cached in the so-called Translation Lookaside Buffer
(\gls{tlb}).


\section{Page table management}
\label{sec:pagetables}
We wrote the following functions to manage the page table:
\begin{itemize}
\item \texttt{pgdir\_walk} is called by most of the other functions to walk the
page table, finding the \gls{pte} corresponding to a given virtual address. It
allocates new levels of the page table as needed using \texttt{page\_alloc}
from the previous section.
\item \texttt{page\_insert} is used to insert a physical page into a page
table at a given virtual address. In other words, it finds a \gls{pte} using
\texttt{pgdir\_walk} and stores the physical address there.
\item \texttt{page\_lookup} finds the physical address of a page, given a
virtual address.
\item \texttt{page\_remove} invalidates a \gls{pte} in a page table.
\end{itemize}
To set up the page table, the kernel allocates pages of physical memory and
uses the newly implemented functions to insert these into the table.

The memory layout of the address space of the kernel is largely up to us.
Figure \ref{memlayout} gives a simplified overview of the layout we opted for.
\begin{figure}
\begin{framed}
\begin{Verbatim}[fontsize=\small]
Virtual memory map:                                  Permissions
                                                     kernel/user
   4 Gig -------->  +------------------------------+
                    :              .               :
                    :              .               :
                    |------------------------------| RW/--
                    |                              | RW/--
                    |      Kernel code, data       | RW/--
                    |                              | RW/--
   KERNBASE, ---->  +------------------------------+ 0xf0000000      
   KSTACKTOP        |     CPU0's Kernel Stack      | RW/--  KSTKSIZE 
                    +------------------------------+                 
                    |     CPU1's Kernel Stack      | RW/--  KSTKSIZE 
                    +------------------------------+                 
                    :              .               :                 
                    :              .               :                 
                    +------------------------------+ 0xef800000
                    |  Cur. Page Table (User R-)   | R-/R-  PTSIZE
   UVPT      ---->  +------------------------------+ 0xef400000
                    |          RO PAGES            | R-/R-  PTSIZE
   UPAGES    ---->  +------------------------------+ 0xef000000
                    :              .               :                 
                    :              .               :                 
   USTACKTOP  --->  +------------------------------+ 0xeebfe000
                    |      Normal User Stack       | RW/RW  PGSIZE
                    :              .               :                 
                    :              .               :                 
                    +------------------------------+
                    :              .               :
                    :              .               :
                    +------------------------------+
                    |     Program code, data       |
   UTEXT -------->  +------------------------------+ 0x00800000
                    :              .               :                 
                    :              .               :                 
                    +------------------------------+ 0x00000000
\end{Verbatim}
\end{framed}
\caption{The virtual address space of the kernel}
\label{memlayout}
\end{figure}
The diagram shows that the kernel code resides starting at virtual address
\texttt{0xf0000000}. The kernel stacks, used by processors when running
kernel-mode code, reside just below, between \texttt{0xefc00000} and
\texttt{0xf0000000}. Further down in the address space, between
\texttt{0xef400000} and \texttt{0xef800000}, we have the User Virtual Page
Table (UVPT) area, which gives user-mode processes read-only access to their
page table, enabling certain exokernel-style programs to work. The stack of
the user-mode program starts at \texttt{0xeebfe000} and grows towards lower
addresses. The user-mode code and data reside near the bottom of the address
space, around \texttt{0x00800000}.

We created a page table according to this layout. We then updated the
\texttt{cr3} register and set a bit in the \texttt{cr0} register to enable the
use of the page table. At this point our kernel had proper virtual memory.
Once user-mode processes have been implemented, virtual memory guarantees that
the processes cannot modify the address space of one another, nor can they
corrupt the kernel.


%%%%% LAB 3 %%%%%
\chapter{User space}

The goal of this lab was to run a user-mode process. This required us to write
infrastructure for managing processes and their metadata. We also modified our
kernel to handle any exceptions generated by user-mode processes. Finally we
implemented a system call mechanism to let processes interact with the kernel.


\section{Managing process metadata}
Each process has some associated information. This includes its state
(running, runnable, blocked, killed), its process ID, parent process ID, page
directory, saved register state, and so on. This metadata is stored in a
struct \texttt{Env}. The process subsystem is similar to the physical page
subsystem; an array holds the \texttt{Env} struct of each process on the
system, and free environments are stored in a linked list. We implemented
functions for creating, initializing, and destroying a process.

\section{ELF loading}
Programs are represented as ELF files. To launch a process, the kernel first
allocates a fresh page table. It then walks over each section in the ELF file,
reads at which virtual address the section should go, allocates corresponding
physical pages, inserts them into the page table, and copies the code or data
into the physical pages.

Note that we have not yet introduced a file system, so it is not immediately
clear where the kernel can find the programs which it should load. To solve
this problem we embed each user-mode program into the kernel as a blob of
binary data. In section \ref{sec:fs} we describe our implementation of a
proper file system for storing programs and data.

\section{Context switching}
Once a program has been loaded, the kernel must perform a context switch to
let the created process run. During a context switch the kernel restores the
saved general-purpose registers using the \texttt{popal} instruction. It also
updates \texttt{cr3} to switch to the new address space. Finally it uses the
\texttt{iret} instruction to restore the saved \texttt{eip}, \texttt{esp},
\texttt{eflags}, and \texttt{cs} registers. This transfers execution to
user-mode code. 

The kernel must also drop its privileges. The lower two bits of the
\texttt{cs} register determine the privilege level of the processor. This was
previously 0, since the kernel runs in ring 0. By setting this to 3 during the
\texttt{iret}, the processor switches to user-mode operation, which is ring 3.
Since instructions such as \texttt{iret} may only be run from privilege level
0, the process cannot simply modify its \texttt{cs} register to increase its
privileges.

At this point our kernel was able to successfully create a new process, load
program code and data into its address space, and let the process run.
Unfortunately the code had no way to give control back to the kernel, so it
simply ran forever, or at least until it triggered an exception or interrupt.
Since this was not yet handled, any exception caused the whole system to
crash.


\section{Handling of exceptions and interrupts}
The processor may trigger an exception while processing an instruction, e.g.,
on a division by zero or illegal memory access. The processor may also
occasionally trigger an interrupt. This often happens for asynchronous
reasons, such as when a key is pressed or a network packet is received.
Interrupts can also be raised with the \texttt{int} instruction; this
mechanism can be used to implement system calls.

The result of an exception or interrupt is that the processor enters kernel
mode through a context switch, running exception- or interrupt-specific
handler code. This handler code is found as follows. Exceptions and interrupts
are numbered. This number specifies an index into a table called the Interrupt
Descriptor Table (IDT), which can be found using the IDT register,
\texttt{idtr}. The IDT entry describes the address of the handler code. 

% As noted, an exception or interrupt triggers a context switch. Therefore the
% processor needs to know on which stack it should save the registers of the
% faulting process. To find this \gls{esp} value, the processor reads the Task
% Register (\gls{tr}) which is an index into the \gls{gdt}. The \gls{gdt} entry
% contains the address of a data structure called the Task State Segment
% (\gls{tss}). The processor reads the new \gls{esp} value from the \gls{tss}.
% 
% To sum up: when an exception or interrupt occurs, the corresponding number is
% looked up in the \gls{idt} to find the new \gls{eip} value. The \gls{tr},
% \gls{gdt} and \gls{tss} are used to find the new \gls{esp} value. The
% processor uses this information to store the registers of the faulting process
% onto the new stack and execute the relevant handler code.


%\section{Handling exceptions and interrupts}
We wrote a common function, \texttt{trap}, which is called whenever an
exception or interrupt occurs. For each exception and interrupt, we filled in
its IDT entry with a small stub which passes the number of the exception
or interrupt as the first argument and calls \texttt{trap}.

The typical result of an exception is that the kernel terminates the running
process. If an exception occurs in kernel mode, the result is a kernel panic,
where the kernel prints an error message and hangs.



\section{Handling of system calls}
A process frequently needs to call kernel code, e.g. to print text onto the
screen or to perform inter-process communication. This is called a system
call. 

One mechanism to implement system calls on the x86 architecture is to use the
\texttt{int} instruction, which triggers an interrupt when executed. We are
free to use any interrupt that is not in use, so we arbitrarily chose number
\texttt{0x30}. Thus a program uses the \texttt{int 0x30} instruction to
perform a system call. We chose a fairly standard ABI; the \texttt{eax}
register holds the system call number, while arguments go in registers
\texttt{ebx}, \texttt{ecx}, etc.

The \texttt{trap} function recognizes interrupt number \texttt{0x30} and calls the
\texttt{syscall} function, which uses a large \texttt{switch} to delegate each
system call to a specific handler.

We wrote kernel system call handlers for input and output of a single
character, as well as for process termination. User-mode programs did not have
a C standard library at this point, so we created one, adding an interface to
the system calls to the library. At this point we were able to run a simple
user-mode program and have it interact with the user. 



%%%%% LAB 4 %%%%%
\chapter{Multiprocessing}
In this lab we added the features necessary for running multiple processes
concurrently and having them interact. 

We implemented a simply round-robin scheduler. The scheduler can preempt the
running process when its time slice is up with the aid of a device called a
LAPIC, which can generate periodic timer interrupts.

We also wrote code to activate any processors beyond the first, letting
processes run truly concurrently. 

We implemented a naive fork mechanism which lets proceses spawn more
processes. We then upgraded its efficiency with a copy-on-write mechanism.

Finally, we added an inter-process communication feature, letting processes
send values and pages of memory to one another.



\section{The process scheduler}
\label{sec:preempt}
The main purpose of a scheduler is to decide which process to schedule in when
the running process is scheduled out. There are many ways to make this
decision, and the scheduler has a great impact on system responsitivity and
efficiency. 

However for simplicity reasons, we opted for a simple round-robin scheduler.
That is, the scheduler keeps a circular queue of all processes. The first
runnable process in the queue is chosen to be scheduled in.

The scheduler also needs to preempt each process when its time slice is used.
To accomplish this, during kernel initialization the kernel asks a device
called the LAPIC to raise a timer interrupt periodically, waiting some fixed
amount of bus cycles between each raised interrupt. Our \texttt{trap} function
recognizes this timer interrupt and reacts by asking the scheduler to schedule
in a new process. Interaction with the LAPIC is described in section
\ref{sec:mpconfig}.



\section{Activating more processors}
\label{sec:moreprocs}
So far the kernel has run on an emulated single-core machine. However a
machine with $n$ processors can be emulated by passing the \texttt{-smp n}
option to QEMU.

When a system with multiple processors boots, the hardware dynamically selects
only a single processor to run. It is called the Bootstrap Processor (BP)
while the remaining processors, if any, are called Application Processors
(APs). It is up the the BP to start up the APs when the system is ready.

We wrote code which starts up the APs. This is accomplished by querying the
LAPIC of the BP, having it send an inter-processor interrupt to each AP. Upon
receiving such an interrupt, the APs start executing code at an address
specified by the AP.

The APs start in 16-bit real mode, just as the BP did. They therefore need to
switch to 32-bit protected mode mode. After doing so, each AP calls into the
scheduler to run a new process.

% source:
% https://stackoverflow.com/questions/14261612/which-core-initializes-first-when-a-system-boots
% which refers to an intel manual



\section{Ensuring mutual exclusion}
With multiple processors running concurrently, all the typical issues of
concurrency arose. Multiple processors could modify kernel data
simultaneously, leading to race conditions.

We prevented this with a trivial but inefficient approach: we added a lock
which must be locked before running any kernel code. Thus only one processor
may run kernel code at a time. Still, user-mode processes can run truly
concurrently.

The kernel lock is a spinlock. It is a global variable. A processor repeatedly
uses the \texttt{lock} and \texttt{xchg} instructions to atomically exchange
the global variable with the value 1. If the old value was zero, the processor
now holds the lock and may enter the kernel. Otherwise it retries.

We added calls to lock and unlock the kernel in the right places. At this
point our kernel was capable of running multiple user-mode processes
concurrently.



\section{Interacting with the LAPIC}
\label{sec:mpconfig}
% source: 
% http://wiki.osdev.org/APIC
% https://en.wikipedia.org/wiki/Advanced_Programmable_Interrupt_Controller

In section \ref{sec:preempt} the scheduler had to ask the LAPIC to generate
timer interrupts for it, and in section \ref{sec:moreprocs} the BP used the
LAPIC to send inter-processor interrupts to the APs. We therefore needed code
to query the LAPIC.

First, however, we explain the acronym. A Programmable Interrupt Controller
(PIC) is a hardware device responsible for managing interrupts for the
processor. For example, if multiple interrupts are generated simultaneously,
the PIC can prioritize the interrupts and deliver them one at a time. When
Intel updated their PIC standard to include new features, the conforming
device was called an Advanced PIC (APIC). The APIC has a component called the
Local APIC (LAPIC) which is local to each processor.

The processor communicates with the LAPIC using Memory-Mapped I/O
(\gls{mmio}). This means that the LAPIC is mapped into memory at a specific,
system-dependent physical address. Reading at certain offsets will correspond
to reading from certain registers in the LAPIC and likewise for writing. Thus
to communicate with the LAPIC, the kernel merely needs to read and write to
certain addresses. The difficult part is figuring out the physical address
where the LAPIC resides.

There are multiple ways to find the LAPIC. For now, our kernel uses the method
described in Intel's multi-processor specification, which we will refer to as
the \gls{mpconfig} method. It involves searching through parts of physical
memory to find a structure called the MP floating pointer structure. This
structure points to a table called the MP configuration table, which contains
the physical address of the \gls{lapic}. 
% TODO: mention that we were given mpconfig code
% TODO: insert reference to specification..



\section{A simple fork mechanism}
So far, every process was directly spawned by the kernel. However, a process
should also be able to spawn processes. We therefore needed a mechanism to let
a process fork. In this section we describe our initial, simple implementation
of fork, and in the following section we improve it by introducing a
copy-on-write mechanism.

Since our kernel is an exokernel, we prefer to keep code outside of kernel
land. We therefore wrote a number of system calls which can be combined to
implement a user-mode fork. Specifically, we wrote the following system calls:
\begin{itemize}
\item	\texttt{sys\_exofork} creates a non-runnable child process with an empty address space.
\item	\texttt{sys\_env\_set\_status} can mark a process as runnable.
\item	\texttt{sys\_page\_alloc} allocates an empty page in the address space
of a process.\footnote{For security reasons, the system calls are coded to
ensure that a process cannot modify pages in unrelated, non-child processes.}
\item	\texttt{sys\_page\_map} maps a page from the current process into a child process.
\item	\texttt{sys\_page\_unmap} unmaps a page from the current process or a child process.
\end{itemize}

Besides these system calls, it is also necessary for a process to have access
to information about the layout of its own address space. This is already the
case; in section \ref{sec:pagetables} we set up the page table of a process
such that part of the address space contains its page table.

To fork, a parent process goes through the following steps:
\begin{itemize}
\item The parent process calls \texttt{sys\_exofork} to create a new child
process with an empty address space which is not initially runnable.
\item The parent walks over its page table, and for each mapped page, it does
the following:
\begin{itemize}
\item The parent uses \texttt{sys\_page\_map} to create a temporary page at a
temporary address.
\item The parent copies the contents of the current page into the temporary
page.
\item The parent uses \texttt{sys\_page\_map} to insert the temporary page
into the address space of the child process at the original address.
\item The parent uses \texttt{sys\_page\_unmap} to remove the temporary page
from its own address space.
\end{itemize}
\item The parent marks the child as runnable using
\texttt{sys\_env\_set\_status}. 
\end{itemize}
At this point the fork is complete, and since the child is marked as runnable,
it will eventually be scheduled in.



\section{Copy-on-write fork}
% TODO: this section needs more polishing, it's a bit hard to understand
% what's going on..
The fork mechanism described in the previous section is slow and
memory-inefficient, because it indiscriminately copies every page of the
parent into the child process. This involves a lot of reading and writing of
RAM, and it means that if the parent used $n$ physical pages of memory, then
after a fork, $2n$ physical pages will be used.

However the same physical page can transparently be mapped into both the
parent and child process, as long as it is never modified. In fact, \emph{all}
the pages in the child process can initially be shared with the parent. It is
only once a write happens that a page must be copied. We have implemented such
a copy-on-write fork mechanism almost entirely in user land, following the
exokernel design philosophy.

To perform as much work in user land as possible, we implemented a mechanism
which lets a process handle its own page faults. By default a page fault will
result in process termination. However we have implemented a system call,
\texttt{sys\_env\_set\_pgfault\_upcall}, which lets a process set a handler
function. Then the kernel will handle a page fault by modifying the saved
\gls{eip} and \gls{esp} registers of the process, pushing the old register
values onto an exception stack, and switching the process back in.

When a process forks, \emph{all} pages are initially shared between the parent
and child process. However writable pages have their writable bit removed from
their page table entry (\gls{pte}). Instead we set another bit which marks the
page as copy-on-write.

When the child or parent process attempts to write to one of the now-shared
pages, a fault will occur since the page is not writable anymore. The kernel
sees that the copy-on-write bit is set, and then it delegates to the
registered user-mode handler. The handler then uses the same method as in the
simple fork implementation to map a new writable page, copy the contents of
the old page onto it, and replace the copy-on-write page with the new writable
page.


\section{Inter-process communication}
Since this project follows the exokernel philosophy, many future features will
reside in user land. Two examples are a file system daemon and a daemon
implementing a network stack. Other processes need an inter-process
communication (IPC) mechanism to make use of these daemons.

We therefore implemented two system calls, \texttt{sys\_ipc\_recv} and
\texttt{sys\_ipc\_try\_send}. When a process calls \texttt{sys\_ipc\_recv} it
will hang, waiting to receive data. \texttt{sys\_ipc\_try\_send} will send
data to a process in a non-blocking fashion. By default, a 32-bit integer is
sent between processes, but for efficiency an extra argument to the system
calls allows the sender to share a full page of memory per system call.



This marks the end of the multiprocessing lab. Our kernel can now run
user-mode processes in a truly concurrent fashion. A scheduler manages the
running processes, preempting them when necessary, and processes can
efficiently fork and communicate via IPC.


%%%%% LAB 5 %%%%%
\chapter{File system}
\label{sec:fs}
The goal of this lab was to implement a custom file system and a user-mode
shell program.

\section{File system design}
Having a file system lets programs store data persistently on disk. It also
provides a storage place for programs, instead of embedding them directly in
the kernel as we have done so far. In this section we describe our design of
a custom file system.\footnote{
To keep our file system simple, we have left out many potential features, such
as file and folder permissions, symbolic and hard links, and timestamps.}

A file system can be thought of as a way to manage how files, folders, and
their metadata are stored on a disk. A raw disk is essentially a portion of
memory which can be read from and written to. It is customary to partition the
memory of a disk into fixed-size blocks. For our file system, the block size
will be 4096 bytes.

One of these blocks is special, since it contains metadata for the file
system. This block is called the superblock. This metadata includes such
things as the disk size and where to find the root folder.

Our file system is laid out as follows. Block 0, is not used by our FS; it is
reserved for the boot loader. Block 1 is the superblock. The next few blocks,
starting at block 2, hold a bitmap which determines whether the remaining
blocks on disk are in use or free. The remaining blocks are used to store the
concrete files and folders.

% TODO: include graphics from https://pdos.csail.mit.edu/6.828/2016/labs/lab5/disk.png

A file is represented as a \texttt{struct file}, which is stored in its own
block. Such a \texttt{file} struct contains metadata, such as the file name
and size. The struct also has 10 pointers to blocks that hold the raw file
data. If the data cannot fit in 10 blocks, the \texttt{file} struct has a
pointer to a block which holds another 1024 pointers to data blocks. Thus our
file system has a maximum file size of $(10+1024)*4096 = 4235264$ bytes, i.e.,
around 4 MB.

% TODO: also this https://pdos.csail.mit.edu/6.828/2016/labs/lab5/file.png

A folder is represented as a \texttt{struct folder}, which is exactly
identical to a \texttt{struct file}, except that the $10+1024$ block pointers
no longer point to raw data, but to other blocks holding \texttt{file} or
\texttt{folder} structs. There is a type flag which allows distinction between
files and folders.

We used a small script to create a raw disk image with an initialized file
system of the described format. The script let us add files, such as sample
programs, to the file system. We then attached this raw disk to QEMU.



\section{File system daemon}
In accord with exokernel design, we let all file system interaction go through
a user-mode process which we call the file system daemon.

The daemon interacts with the disk using \gls{pio}. However a normal user-mode
process cannot use the \texttt{inb} and \texttt{outb} instructions to perform
\gls{pio}, so our kernel needs to give the daemon I/O privileges. It does so
by setting a bit in the \texttt{eflags} register while spawning the daemon.

The disk knows nothing of the file system which is stored upon it. The daemon
can merely read a sector of the disk at a time. It is therefore up to the
daemon to implement reading and writing of files and folders according to the
file system specification. We wrote the necessary functions for interacting
with the file system.

The file system daemon spends its time looping, waiting for other processes to
contact it via IPC. Processes can send requests to open, read, write,
and stat files. The IPC details are hidden inside the user-mode C standard
library, giving processes the usual interface with functions such as
\texttt{open}, \texttt{read} and \texttt{write}.

For sake of illustration, the following happens when a process wants to open a
file:
\begin{itemize}
\item The process calls the \texttt{open} library function.
\item The library uses IPC to contact the file system daemon.
\item The daemon reads the superblock to find the block of the root folder.
\item The daemon follows pointers from folder to folder according to the given
path.
\item When the path has been traversed, the daemon has found the block that
contains the \texttt{file} struct.
\item The offset into the disk that holds the \texttt{file} struct is stored
in a file descriptor.
\item File descriptors are numbered; the daemon returns this number as the
result of the IPC call.
\end{itemize}
When the process subsequently reads from the file descriptor, the daemon
uses the pointers in the \texttt{file} struct to find the raw data and return
it via IPC.

We have also implemented a block caching system to improve the efficiency of
the file system daemon. When a block is first read, its contents are stored in
RAM, and subsequent reads do not need to interact with the disk until the
cache entry is invalidated through a write.


\section{Shell}
To test the interaction between programs and the disk, we implemented a
simplistic shell which allows loading and execution of programs from disk. The
shell allows reads from and writes to the disk through input redirection
(\texttt{<}) and output redirection (\texttt{>}). The shell additionally
supports piping of the output of one program into another program. A small
program, "cat", lets the user read files from the disk. A sample interaction
is shown in figure \ref{fig:shellinteraction}.

\begin{figure}[h]
\begin{framed}
\begin{Verbatim}[fontsize=\small]
$ ls
cat
echo
ls
sh
$ echo "Hello, OS." > motd
$ cat motd
"Hello, OS." 
\end{Verbatim}
\end{framed}
\caption{A sample interaction with the shell}
\label{fig:shellinteraction}
\end{figure}



%%%%% LAB 6 %%%%%
\chapter{Networking}
The goal of this lab was to connect our kernel to the internet by writing a
network card driver.

\section{The network card}
The operating system connects to a network using a network card, whose task it
is to send and receive packets by interacting with the physical layer. QEMU
emulates the Intel e1000 network card, which we therefore targeted. We added
the card to our emulated machine with the "-net nic,model=e1000" option to
QEMU, which creates a virtual router at IP 10.0.2.2 and assigns the guest an
IP of 10.0.2.15. This let us hardcode the the IP address rather than having to
implement a DHCP client or similar.

% %TODO: insert reference to the manual and a proper name..
The e1000 manual describes in detail the internals of the card, as well as the
steps a driver must take to initialize the card, read incoming packets, and
transmit outgoing packets. 

The e1000 maintains two circular packet queues: a transmit queue and a receive
queue. To send a packet, the kernel adds it to the transmit queue, which the
network card periodically drains. When a packet arrives on the wire, the
network card adds it to the receive queue, which the kernel periodically
drains.

The items in the queues are not raw packets, but rather small descriptor
structs, each of which describes an area of physical memory which can hold a
packet. The queues are implemented as arrays, with each queue having a head
and tail register to keep track of the head and tail of the queue.



\section{The network card driver}
We wrote an e1000 network card driver using the information from the manual.
Our kernel uses \gls{pio} to scan the PCI bus, iterating over each connected
device until it finds one whose vendor and device ID indicate it being an
e1000 network card. The PCI interface supplies a physical address where the
driver can use \gls{mmio} to communicate with the e1000.

With communication possible, the first task of the driver was to initialize
the network card. The driver allocates the arrays that make up the packet
queues and fills these with valid descriptors. It zeroes out the head and tail
registers.

Once our driver has initialized the card, it writes into a register to enable
it. It is then possible to transmit and receive packets. Our driver implements
the code for taking packets from the receive queue and inserting packets into
the transmit queue. User-mode programs have access to this driver
functionality through two system calls, \texttt{sys\_receive} and
\texttt{sys\_transmit}, respectively.

If the transmit queue ever becomes full due to the card draining it too
slowly, it is up to the driver what should happen. For simplicity we have
chosen to simply drop the packet if the queue is full. This is not a problem
because the higher-level protocols are resistant to packet loss. Alternatively
the driver could have blocked, waiting for space in the queue. If, on the
other hand, the receive queue is full, the e1000 is designed to simply drop
further packets.


% TODO: lesson: writing drivers is tricky. (debugging: QEMU source modification)


\section{The network daemon}
With the driver written, our kernel was capable of transmitting and receiving
packets. However most processes only know of the data they wish to send; they
are not capable of constructing packets.

We therefore needed a TCP/IP stack. Allegedly, writing such a network stack
from scratch is an immense task. We therefore used the open-source stack lwIP
("lightweight IP"). lwIP acts as a black box for our purposes, taking raw data
as input, and producing packets as output.

Rather than adding lwIP to the kernel, we embedded it in a network daemon. The
daemon is responsible for managing sockets, just as the file system daemon was
responsible for managing file descriptors. The network daemon takes in
requests to send data via IPC, produces packets, and hands these over to the
operating system via the \texttt{sys\_transmit} system call. Likewise, it
takes in requests to receive data and uses \texttt{sys\_receive}, parses the
resulting packets, and hands over the data to the requesting process.

The IPC communication is hidden away in the C standard library, such that
user-mode programs have access to the familiar \texttt{connect}, \texttt{send}
and \texttt{recv} interface.


\section{Web server}
To test the new functionality, we wrote a simple web server which can serve
files from the file system over HTTP. The web server uses the network daemon
to accept incoming connections and receive HTTP requests. It parses each
request, queries the file system daemon to retrieve the requested file
contents, and uses the network daemon to send back the HTTP response.

We configured QEMU to forward requests at port 80 to the emulated machine. At
this point we were able to point a browser at the machine and be served a
working web page.

% TODO: maybe insert a screenshot.






%%%%% GRAPHICS LAB %%%%%


\chapter{Graphics}
\label{sec:graphics}

In this lab we implemented a graphical user interface to our operating system.
The end result is that each application is given its own graphical window on
the screen, and the user can interact with applications using the mouse and
keyboard.

The pixels on the screen are represented in a data structure called the Linear
Frame Buffer (LFB), which the kernel must get access to by querying the BIOS.
The kernel can draw pixels on the screen by writing to the LFB. We wrote a
graphics library to draw pixels and more complex shapes.

We then designed a graphics stack, which involved deciding how input and
output events should flow through the system, from kernel to user application.
We decided upon a design with a central display server which is responsible
for rendering all of the screen and which acts as an intermediary for input
and output events.

Finally we added a few sample graphical applications to demonstrate the new
features.


\section{Mapping the Linear Frame Buffer}
To initialize graphics, the operating system must query the BIOS to select a
video mode, which is a description of the width, height and depth of the
screen. Once a video mode is selected, the BIOS maps a crucial data structure
into memory, which is called the Linear Frame Buffer (LFB).

The LFB represents the pixels of the screen; it can be thought of as a
two-dimensional array, where each value is a 32-bit integer representing the
RGB value of a pixel on the screen. Thus to write a pixel to the screen, the
kernel must simply calculate the correct offset into the LFB and write a
32-bit integer there.

The kernel queries the BIOS to set a video mode, which is done using the
\texttt{int 0x10} instruction. This transfers execution to the BIOS. However
since the BIOS is written as 16-bit real mode code, our kernel must switch the
processor back to 16-bit real mode before it can issue the interrupt.
We therefore wrote assembly code which switches the processor mode,
queries the BIOS to enumerate the valid video modes, and selects one with a
satisfactory resolution.

At this point our kernel was able to draw pixels on the screen by writing to
the LFB. However the kernel lacked the ability to draw more complex shapes.


\section{Graphics library}
We used the ability to draw a single pixel as a building block, writing
functions that can draw more complex shapes, such as straight lines,
rectangles, and windows. We also implemented font rendering.

Since this functionality is needed by many different applications, we put it
inside a graphics library which is available to user space applications.


\section{Graphics stack design}
% <--- TODO got to here.
To design our graphics stack, we first read up on how graphics work in common
operating systems and used this as a basis for our design. We decided to
have a central privileged process, called the display server, which is
responsible for most of the work.

The display server is responsible for spawning other graphical applications,
assigning a portion of the screen to each application. The display server is
the only process with access to the LFB, so it is responsible for
drawing all pixels to the screen. 

Each graphical application is spawned by the display server. When this
happens, the two set up an area of shared memory through \gls{ipc}. This
memory is called the canvas of the application. The application can write
pixels into the canvas, and the display server will periodically read the
pixels from the canvas and write them to the LFB.

Graphical applications wait in a so-called event loop; they continuously wait
for input events to arrive from the display server. When an input arrives, the
application handles it and can make changes to its canvas on this basis.

The kernel keeps a queue for input events. When raw input packets arrive from
the mouse or keyboard, a driver handles these packets by putting them into an
event queue. The display server periodically drains this queue through a
system call. It then forwards each event to the appropriate application.

A graphics library provides common functionality needed by the user
applications and the display server. The library provides functions for
rendering rectangles, straight lines, fonts, and windows.

The following diagram shows an overview of the different components and how
they interact with each other:
% TODO insert a diagram.


Thus imagine that a user sits in front of the machine with a terminal emulator
application open and active. If the user presses the 'A' key, the following
series of events will occur: 
\begin{itemize}
\item The user presses the 'A' key.
\item The keyboard generates an interrupt to signal to the kernel that input
is available.
\item The kernel keyboard driver reads the pressed key using \gls{pio}.
\item The driver puts the event into the events queue.
\item The display server eventually drains the events queue and finds the key
press event.
\item The display server finds that the terminal application is active and
therefore forwards the event to it using \gls{ipc}.
\item The terminal application receives the event and adds the 'A' to a buffer
which holds the current input of the user. This buffer will eventually be used
to run a command once the user presses the enter key.
\item The terminal application also wants to display the 'A' on the screen, so
that the user can see what is written so far. The application therefore asks
the graphics library to render an 'A' on its canvas using the default font.
\item The display server writes the pixels from the canvas into part of the
LFB.
\item The user sees the 'A' appear on screen.
\end{itemize}


% TODO: tie these sections together better

\section{Details of the display server}
The display server needs to have write access to the LFB, which is
mapped at a physical address by the BIOS. To accomplish this, the display
server uses a new system call, \texttt{sys\_map\_lfb}, which causes the kernel
to add the LFB to the page table of the display server.
% TODO: rephrase this to "we could make the kernel write to LFB, but instead
% for effiency we map it directly via page tables"
% then the title of this can be "effiency"..

It is highly important that the display server is efficient; if it runs too
slowly, the system will have a low frame rate, and interaction will feel
choppy. We had to carefully optimize the code for the display server before we
got an acceptable frame rate inside QEMU.

One of the optimizations is the following. The display server takes the
content of each canvas and writes it into the LFB. However if
canvases overlap, there is no need to first write the lower canvas into the
\gls{lfb}, and then immediately overwrite it with the canvas that is on top.
We therefore introduced a buffer held in RAM. The display server first
constructs the final canvas in this buffer, and then copies each pixel once
and for all into the LFB.

% TODO: write about this: - uses simple data structures (linked lists)


\section{Sample graphical applications}
We have written two sample graphical applications. One is a simple terminal
emulator. The other is a simple paint program.

% TODO: maybe describe these in some more detail?

% TODO: fit in details about PS/2 mouse driver somewhere.








%%%%% HARDWARE LAB %%%%%
\chapter{Hardware}
Up to this point, our kernel had always been running in the QEMU emulator. In
this lab, our goal was to get it running on real hardware, specifically a
Packard Bell Dot S netbook. The road there was paved with surprisingly many
complications. Most of the issues were caused by the fact that QEMU emulates
different hardware than that of the netbook. There were also instances where
QEMU did not emulate certain aspects of a machine faithfully, causing latent
bugs to surface on real hardware.


\section{Booting from USB}
The build process of our kernel produces a raw disk image which QEMU will
boot. The first block of this image is the boot loader, which the BIOS loads
and transfers execution to as described in section \ref{sec:lab1}. Since QEMU
will boot from this image, we figured that so would the netbook. We put the
raw disk image on a USB drive and had the netbook boot from it. However the
netbook did not recognize the USB as a bootable medium.

It turns out that a USB drive must have a valid data structure called a Master
Boot Record (MBR) in its first block, otherwise most machines will not
recognize the drive as bootable. The MBR must also contain a valid data
structure called the BIOS Parameter Block (BPB).

The reason for this is historical; when USB technology was new, there was
disagreement on whether a USB drive should be a raw storage drive or a
bootable medium. Initially the user could decide through a BIOS setting. But
for usability reasons, manufacturers eventually instead used heuristics to
detect whether a USB is bootable or not. These heuristics are based on the
MBR and BPB.

Once we added a valid MBR and BPB to our boot loader, the netbook booted.
However the machine got stuck at some point during the boot loader code.


\section{No serial connection}
Previously the kernel printed debugging information through a serial
connection which QEMU emulated. However we did not have the hardware necessary
to set up a serial connection to the netbook. We therefore had no output,
which made it difficult to determine why the boot loader was hanging.

We therefore had to write code which outputs text to the screen instead.
Before a video mode has been set as described in section \ref{lab:graphics},
the machine is in text mode. In text mode, by convention a buffer at physical
address \texttt{0xb8000} represents the text on the screen; we can write characters
directly into this buffer to show text on screen.


\section{No USB driver}
With the ability to print text to the screen, we figured out why the boot
loader was hanging. Our boot loader is naively coded such that it loads the
kernel from a disk connected with an ATA connection. But currently the kernel
resides on a USB drive, which is a completely different interface.

To continue, we needed to write a USB disk driver for the boot loader. However
the boot loader is still constrained to 512 bytes, since it is loaded from the
first sector of disk by the BIOS. Fitting a USB driver there is tricky.

We therefore opted for a different approach. We booted from USB into a Linux
distribution, and used that to overwrite the hard drive of the netbook with
our raw kernel image. Now we can boot from a proper hard drive rather than
USB.

This approach had the major downside of being slow; we had to boot the weak
netbook into a Linux distribution and enter several commands manually every
time we wanted to test a new iteration of our kernel. This slowed development
speed down significantly.

Additionally, the boot loader continued to get stuck.


\section{No SATA support}
After more debugging we determined the problem: the hard drive in the netbook
uses a SATA connection, while the machine emulated by QEMU used ATA. Thus
reading the kernel from disk was failing.

Fortunately, ATA and SATA are almost identical. ATA uses \gls{pio} to
communicate with the disk on fixed ports. SATA uses the same protocol; the
only difference is that the ports are machine-specific and must be found with
\gls{pci}.

So we tried to add PCI support to our boot loader but ran out of space -- we
only had 512 bytes to work with, after all. As a temporary workaround, we
booted into a Linux distribution and used the "lspci" tool to figure out the
SATA I/O ports and hardcoded them. 

With that, the boot loader finally succeeded in loading the kernel, which
promptly broke; the scheduler was failing to switch in new applications.


\section{A better boot loader}
We attempted to debug the scheduler, but our development process was too slow
and unwieldy to get anywhere; as mentioned, every time we made a change to the
kernel code we had to boot into a Linux distribution and enter commands to
write the raw kernel image to the hard disk. We needed to get around this;
ideally we would simply insert a USB drive and immediately boot into our
kernel. 

To facilitate booting from USB, we replaced our boot loader with a better one.
GRUB is the gold standard for boot loaders; it supports booting from various
hard drive types, USB, and even booting via an ethernet connection. After
integrating GRUB into the project, we were finally able to boot directly from
USB, and development sped up significantly.

Our custom file system assumes that the boot loader will fit into the first
512 bytes, with the superblock residing in the following sector. However as
GRUB uses multiple stages, it needs much more space. We therefore had to
modify our file system such that the first 32 MB are reserved for GRUB, and
the superblock resides thereafter.


\section{Finding the LAPIC}
Usually the LAPIC generates timer interrupts periodically, and on such an
interrupt the scheduler switches in a new process. However no timer interrupts
were generated, and thus only one process got any processing time. 

In section \ref{sec:mpconfig} we described how our kernel uses the so-called
\gls{mpconfig} method to find the LAPIC; that is, it looks for an MP
configuration table in memory to find the physical address of the LAPIC. This
physical address is used to communicate with the LAPIC, asking it to generate
timer interrupts. However our kernel failed to find these MP configuration
tables. They were simply not present in the memory of the netbook.

It turns out that the \gls{mpconfig} method is outdated and unsupported by
modern hardware. We therefore had to implement a different way to find the
LAPIC.

We leave out the details of the method, but in essence newer machines contain
so-called ACPI tables. One of these is the APIC table, which contains the
physical addres of the LAPIC. The ACPI tables can be found by scanning a
region of memory for a data structure called the RSDP, which points at another
data structure, the RSDT, which finally points at the ACPI tables.

After implementing this finding and parsing of ACPI tables, the kernel found
the LAPIC and timer interrupts were generated, letting the scheduler work as
intended.

Then another bug was uncovered.


\section{A page table bug}
Our kernel was behaving oddly; modifying a \gls{pte} seemed to have no effect.
After the modification, writing to memory at the virtual address still
affected memory at the old physical address rather than the new one.

We figured that this must be related to the \gls{tlb}, since this seemed to be
a caching issue. After much searching we learned that the \texttt{invlpg}
instruction must be used to invalidate a \gls{tlb} entry after a \gls{pte} has
been modified. Otherwise the outdated cached \gls{tlb} entry will continue to
be used until it is evicted.

Interestingly, this bug never surfaced while running the kernel in QEMU. We
assume that QEMU does not faithfully emulate the TLB for performance
reasons.\footnote{VirtualBox behaved similarly to QEMU; we therefore believe
that this quirk can be used as a means to detect virtualization/emulation.}


\section{Broken mouse driver}
At this point the kernel successfully booted into a graphical interface.
However the mouse was behaving oddly, jumping around when moved. The PS/2
mouse driver we wrote during the graphics lab was at fault; the driver uses
\gls{pio} to read from the mouse. Before each \texttt{inb} instruction, it is
necessary to wait for the mouse to signal that it has sent another byte of
data. The driver did not do this. However in QEMU these waits were not
necessary; thus this was another instance of QEMU not emulating hardware
perfectly.


\section{The final result}
With all the changes and fixes described so far, the kernel finally ran on our
netbook.
% TODO: include a photo?




\chapter{Future work}
% TODO: we should talk about future work.
% - security: ASLR, probably many other things (it's not even a multi-user
% system yet)
% - capabilities?
% - basic programs like an editor, browser, etc.
% - better windowing system (faster algorithms, better fonts, z-coordinate..)
% - could get a better scheduler
% - could have true kernel concurrency
% - better file system (timestamps, ...)
% - more efficient display server
% - more drivers.... e.g. network cards.


% TODO: I moved this paragraph here from the scheduler.
% Our scheduler has much room for improval. Round-robin scheduling is not as
% performant as more advanced algorithms. Additionally, it does not allow
% adjustment of process priorities. Scheduling being a linear-time operation is
% potentially worrying. 
% 
% A final issue is that the time slice assigned to each process is always of the
% same duration. It would be useful to assign shorter time slices and more
% frequent schedulings to processes which need to feel responsive (such as
% graphics-based processes introduced later) and longer time slices to processes
% that perform heavy computations.

\chapter{Conclusion}
- conclusion

% TODO: we could include screenshots to make things more concrete
% TODO: include (a link to) the source code somehow



% TODO: add references (some links are in writeup txt files)




% TODO: question: is it a good idea to have "mistakes made" and "lessons
% learned"? If so, how do we do this without it sounding too informal?
\begin{framed}
\makebox[\textwidth]{Lesson learned: foo} \\
Maybe we can use something like this to describe the various lessons we
learned...
\end{framed}


% TODO: we should have an appendix that describes how to get and run the code.

\begin{thebibliography}{1}

% use like this: \cite{sigar1, sigar2}

% \bibitem{proc}
% Procyon,
% \\\texttt{http://bitbucket.org/mstrobel/procyon}
% 
% \bibitem{em} 
% Exam Monitor website.
% \\\texttt{http://exammonitor.dk}
% 
% \bibitem{jnlp}
% Exam Monitor JNLP file.
% \\\texttt{http://login.exammonitor.dk/exam.jnlp}
% 
% \bibitem{sigar1}
% Hyperic SIGAR website.
% \\\texttt{https://support.hyperic.com/display/SIGAR/Home}
% 
% \bibitem{sigar2}
% libsigar at GitHub.
% \\\texttt{https://github.com/hyperic/sigar}
% 
% \bibitem{sigarapache}
% The license file of SIGAR.
% \\\texttt{https://github.com/hyperic/sigar/blob/master/LICENSE}
% 
% \bibitem{agents}
% The package java.lang.instrument.
% \\\texttt{https://docs.oracle.com/javase/7/docs/api/java/lang/instrument/package-summary.html}
% 
% \bibitem{javassist}
% Javassist website.
% \\\texttt{http://www.javassist.org}
% 
% \bibitem{openjdk}
% OpenJDK website.
% \\\texttt{http://openjdk.java.net/}
% 
% \bibitem{webportal}
% Exam Monitor web portal.
% \\\texttt{https://sdu.exammonitor.dk/myexams.php}
% 
% \bibitem{x86harmful} 
% Joanna Rutkowska. Intel x86 considered harmful. 2015.
%  
% \bibitem{sokintro} 
% Bhushan Jain, Mirza Basim Baig, Dongli Zhang, Donald E. Porter, and Radu Sion.
% SoK: Introspections on Trust and the Semantic Gap. 2014.
% 
% \bibitem{appshield} 
% Yueqiang Cheng, Xuhua Ding, Robert H. Deng. AppShield: Protecting Applications
% against Untrusted Operating System. 2013.
% 
% \bibitem{towardsappsec} 
% Dan R. K. Ports, Tal Garfinkel. Towards Application Security on Untrusted
% Operating Systems. 2008.
% 
% \bibitem{haven} 
% Andrew Baumann, Marcus Peinado, and Galen Hunt. Shielding Applications from an
% Untrusted Cloud with Haven. In \emph{Proceedings of the 11th USENIX Symposium
% on Operating Systems Design and Implementation.} 2014.
% 
% \bibitem{sokhiee} 
% Fengwei Zhang, Hongwei Zhang. SoK: A Study of Using Hardware-assisted Isolated
% Execution Environments for Security. In \emph{Hardware and Architectural
% Support for Security and Privacy (HASP '16)}. 2016.
% 
% \bibitem{sgxexplained} 
% Victor Costan and Srinivas Devadas. Intel SGX Explained. 2016.
% 
% \bibitem{malwareinenclaves} 
% Jeroen van Prooijen. The Design of Malware on Modern Hardware: Malware Inside
% Intel SGX Enclaves. 2016.
% 
% \bibitem{enclavesinpractice} 
% JP Aumasson, Luis Merino. SGX Secure Enclaves in Practice: Security and Crypto
% Review. Presented at \emph{Black Hat USA 2016}.
% 
% \bibitem{opensgx} 
% Prerit Jain, Soham Desai, Seongmin Kim, Ming-Wei Shih, JaeHyuk Lee,
% Changho Choi, Youjung Shin, Taesoo Kim, Brent Byunghoon Kang, Dongsu Han.
% OpenSGX: An Open Platform for SGX Research. 2016.
% 
% \bibitem{iago} 
% Stephen Checkoway, Hovav Shacham. Iago Attacks: Why the System Call
% API is a Bad Untrusted RPC Interface. 2013.

\end{thebibliography}

% TODO: maybe move this elsewhere?
\printglossary[title=Abbreviations]



\end{document}

