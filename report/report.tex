\documentclass{article}
\usepackage[utf8]{inputenc}
\usepackage{graphicx}
\usepackage{amsfonts}
\usepackage{amsmath}
\usepackage{framed}
\usepackage{hyperref}
\usepackage{fancyvrb}
\usepackage{color}
\usepackage{bm}
\usepackage{listings}
\usepackage{soul,xcolor}
\usepackage[htt]{hyphenat} % allows breaks inside texttt
\usepackage[acronym, toc]{glossaries}
\setstcolor{red}

\usepackage{xcolor}

% this makes links prettier
% TODO: find a nicer way to indicate that text is a link.
\hypersetup{
    colorlinks,
    linkcolor={blue!30!black},
    citecolor={blue!30!black},
    urlcolor={blue!30!black}
}

\lstset{language=C}
\lstset{basicstyle=\ttfamily\footnotesize}

\newglossaryentry{elf}{
	name=ELF,
	description={Executable and Linkable Format; a file format used to store
	executable programs. An ELF file describes the address at which each
	section of program code or data should be loaded.}
}

\newglossaryentry{cr0}{
	name=\texttt{cr0},
	description={Control Register 0. It holds bits which determine how the
	processor operates. It e.g. determines whether paging and protected mode
	are enabled.}
}

\newglossaryentry{cr3}{
	name=\texttt{cr3},
	description={Control Register 3. It points to the current page table.}
}

\newglossaryentry{esp}{
	name=\texttt{esp},
	description={Extended Stack Pointer register.}
}

\newglossaryentry{eip}{
	name=\texttt{eip},
	description={Extended Instruction Pointer; a register which holds the
	address of the next instruction to be executed.}
}

\newglossaryentry{eflags}{
	name=\texttt{eflags},
	description={A register which holds various flags that mostly reflect the
	properties of the most recently executed instruction. For example the
	signed flag is set during a subtraction whose result is below zero.}
}

\newglossaryentry{cs}{
	name=\texttt{cs},
	description={Code Segment selector register; it holds an index into the
	\gls{gdt}. Its lower two bits determine the \gls{cpl}.}
}

\newglossaryentry{cpl}{
	name=CPL,
	description={Current Privilege Level; the current ring in which the
	processor is executing. CPL=3 means ring, 3, i.e., user-mode, while CPL=0
	means ring 0, i.e. kernel-mode.}
}

\newglossaryentry{gdt}{
	name=GDT,
	description={Global Descriptor Table; a table which holds descriptors,
	each of which describes a segment of memory and its permissions.}
}

\newglossaryentry{pte}{
	name=PTE,
	description={Page Table Entry; a 32-bit integer which stores a physical
	address to which a virtual address maps, as well as some status bits.}
}
\newglossaryentry{mmu}{
	name=MMU,
	description={Memory Management Unit.}
}
\newglossaryentry{tlb}{
	name=TLB,
	description={Translation Lookaside Buffer.}
}

\newglossaryentry{pde}{
	name=PDE,
	description={Page Directory Entry; a 32-bit integer which stores a
	physical address of a second-level node in the page table, as well as some
	status bits.}
}

\newglossaryentry{idt}{
	name=IDT,
	description={Interrupt Descriptor Table; a table which describes the
	address of the handler that should be run when a given interrupt or
	exception is triggered.}
}

\newglossaryentry{idtr}{
	name=IDTR,
	description={A register which holds the physical address of the \gls{idt}.}
}

\newglossaryentry{tr}{
	name=TR,
	description={Task Register; a register which is used as an index into the
	\gls{gdt} to find the \gls{tss}.}
}

\newglossaryentry{tss}{
	name=TSS,
	description={Task State Segment; a data structure which, among other
	things, determines the \gls{esp}-value used during a context switch
	triggered by an exception or interrupt.}
}

\newglossaryentry{pic}{
	name=PIC,
	description={Programmable Interrupt Controller; a hardware device which
	orders interrupts before delivering them to the processor}
}

\newglossaryentry{apic}{
	name=APIC,
	description={Advanced \gls{pic}.}
}

\newglossaryentry{lapic}{
	name=LAPIC,
	description={Local \gls{apic}. The LAPIC is the processor-local component
	of the \gls{apic}. Modern systems have one LAPIC per processor.}
}

\newglossaryentry{mmio}{
	name=MMIO,
	description={Memory-Mapped I/O. If you communicate with a device using
	MMIO, it means that the registers of the device are mapped into memory at
	some address, and so communication happens by reading from or writing to
	memory.}
}

\newglossaryentry{mpconfig}{
	name=mpconfig,
	description={mpconfig is a method for finding information about multiple
	processors described in Intel's Multi-processor specification.}
}

\newglossaryentry{bp}{
	name=BP,
	description={Bootstrap Processor; the first, and initially only, processor
	that runs when a system boots.}
}

\newglossaryentry{ap}{
	name=AP,
	description={Application Processor; any processor which is not a \gls{bp}.
	The APs only run once they are started by the \gls{bp}.}
}

\newglossaryentry{ipi}{
	name=IPI,
	description={Inter-Processor Interrupt. An interrupt sent by one processor
	to another using the \gls{lapic}.}
}

\newglossaryentry{pio}{
	name=PIO,
	description={Programmed Input/Output. A way to read and write data from/to
	disk (or another device) using instructions such as \texttt{inb} and
	\texttt{outb}. Often an alternative to \gls{mmio}.}
}

\newglossaryentry{ipc}{
	name=IPC,
	description={Inter-Process Communication. A mechanism which lets processes
	communicate.}
}

\newglossaryentry{pci}{
	name=PCI,
	description={Peripheral Component Interconnect. A type of bus to which
	devices, such as the network card, can be connected.}
}

\makeglossaries



\title{
\line(1,0){250}\\
\Large \bfseries
TODO: write a \\
nice title here
\line(1,0){250}
}
\author{Thomas Hybel}
\date{Aarhus University \\ October 2017}

\begin{document}
\pagenumbering{roman}
\maketitle

\begin{abstract} 
\noindent 
TODO: write an abstract here.
\end{abstract}
\newpage

\tableofcontents
\newpage
\pagenumbering{arabic}



%%%%%%%%%%%%%%%%%%%%%%%%%%%%%%%%%%%%%%%%%%%%%%%%%%%%%%%%%%%%%%%%%%%%%%%%%%%%%%

\section{Introduction}


- introduction
	- what are we trying to do
	- how are we doing it
		- following the course from MIT
	- why are we doing it
	- overview of the coming chapters

% TODO: write about our development environment, i.e., that we're using QEMU.

% TODO: write that you can use "git checkout lab1" to go back to the end of
% lab1, etc.
% TODO: say it's an exokernel, talk about what that means for our later
% design
% TODO: refer to the abbreviations list.

% TODO: read each of the sections' introductions and see whether they make
% sense when read alone in sequence
% TODO: read over the report watching for confusing terms, then add them to
% the glossary (e.g. "real mode")
% TODO: note that the goal was to learn, not to create the most user-friendly
% or efficient system. So we e.g. created our own boot loader initially
% instead of using GRUB, and we created a custom file system instead of
% implementing a common one.

% TODO: we haven't talked about ATA yet.

% TODO: did we describe the console subsystem which lets us do i/o initially?


% TODO: in the end we should browse through the kernel code and see if there's
% any of it we didn't describe anywhere


%%%%% LAB 1 %%%%%

\section{Lab 1}

In this lab we wrote initialization code for our operating system. This code
sets up a rudimentary page table and switches the processor from 16-bit real
mode to 32-bit protected mode. We also wrote a boot loader, which is a small
program that loads the main kernel from disk and transfers control to it.

\subsection{The boot process}
To understand the purpose of a boot loader, it is first necessary to have an
overview of the process which an x86 machine goes through upon startup.

% source: https://en.wikipedia.org/wiki/BIOS
When an x86 machine starts, its BIOS code is run. The BIOS initializes some of
the system's hardware components (e.g., keyboard, graphics card, and hard
drive). The BIOS will then load one sector (512 bytes) from the boot medium
into memory at a hard-coded address (0x7C00). Once this first sector is loaded
into memory, the BIOS will transfer execution to the loaded code.

The first sector will typically contain a small program known as the boot
loader. Its purpose is to load the main kernel from disk and transfer
execution to it.

Before the boot loader loads the kernel, we first run some initialization code
which sets up a more comfortable environment for the boot loader and kernel to
work in.



\subsection{Initialization code}
When the BIOS jumps into our code, the processor is running in 16-bit real
mode. However the code produced by a modern compiler expects to run in 32-bit
protected mode. We therefore needed to write code which performs this
switching of processor modes.

The main difference between real mode and protected mode is how address
resolution is performed. In real mode, accessing a physical address is done
using a segment selector register and a general-purpose register. In
contast, protected mode allows (but does not require) the use of
\emph{paging}, i.e., the mapping from virtual from physical addresses using a
page table.

Since we want our operating system to use virtual memory, we certainly want to
switch to protected mode. We do not immediately enable paging, however.

To switch to protected mode, a bit needs to be set in the control register
\gls{cr0}. We do so with the following assembly code:
\begin{verbatim}
# protected mode enable flag
.set CR0_PE_ON, 0x1 
movl    %cr0, %eax
orl     $CR0_PE_ON, %eax
movl    %eax, %cr0
\end{verbatim}

To switch to 32-bit mode, we must update the \gls{cs} (code
segment) register. This register is an offset into a table called the Global
Descriptor Table (\gls{gdt}) which contains a number of descriptors. Each
descriptor has a bit which determines whether the chosen segment is a 32-bit
or a 16-bit segment. We use the \texttt{ljmp} instruction to update the
\gls{cs} register.
% TODO: possibly elaborate more on the GDT.

Once our code has switched to 32-bit protected mode, it executes the boot
loader. At this point the processor is in a state where it can execute
compiled C code, which means that we no longer need to hand-write assembly.

The file \texttt{boot/boot.S} contains the code that performs the described
tasks.



\subsection{The boot loader}
It is the task of the boot loader to load the main kernel and transfer
execution to it. The boot loader must do this using no more than 512 bytes of
code, minus the bytes used by the initialization code.

Our kernel will be compiled into an \gls{elf} file. The boot loader will parse
this \gls{elf} file, using the information therein to load the code and data
of the kernel to the right physical addresses.

% TODO: possibly include information about the ELF file format here.

Once the boot loader has finished loading the kernel, it determines the entry
point address from the \gls{elf} file and jumps there, transferring execution to the
kernel.

\subsection{Minimal kernel code}
At this point our kernel can finally run. However it does not yet have much
functionality. Unlike user-mode programs, the kernel does not have access to a
C standard library, unless we write one ourselves.


% TODO: document that we were given the serial connection code via the MIT
% course
As a start, we would like to have the ability to input and output text.
Our kernel can use the \texttt{inb} and \texttt{outb} instructions to
communicate with the outside world via a serial connection. We used this to
write out a message and confirm that the kernel runs.

Using instructions such as \texttt{inb} and \texttt{outb} to communicate with
an input/output device is known as Programmed Input/Output (\gls{pio}). It is a
technique we will be using throughout our operating system development.


% TODO: should note that our boot loader isn't actually used in the end.




%%%%% LAB 2 %%%%%

\section{Lab 2}

The goal of this lab was to write code which sets up the page table, so that
our operating system can use virtual memory. Before we could set up the page
table, we first needed to implement a subsystem which manages the physical
memory of the system.

\subsection{Physical page allocation}
A given system has a limited amount of physical memory, depending on how much
RAM the machine has. This memory is split up into a number of pages. On x86, a
page is 4096 bytes of memory. In hex, 4096 is 0x1000, which means that pages
are always aligned on 0x1000-byte boundaries.

% sources: https://en.wikipedia.org/wiki/Nonvolatile_BIOS_memory
% http://wiki.osdev.org/CMOS
To find out how much memory is available on the system, we query a memory area
called the CMOS. The CMOS is an area of memory which holds the amount of
system RAM. We can read from the CMOS using \gls{pio} to figure out how many pages
of physical memory are available to the kernel.

For each page, the kernel must keep track of whether it is in use or not, and
if so, how many references there are to the given page. To accomplish this,
metadata about each page is stored in a \texttt{PageInfo} struct. For example,
the \texttt{PageInfo} struct holds the reference count of each page. The
kernel has an array of \texttt{PageInfo} structs, called \texttt{pages}.
Each entry of the \texttt{pages} array directly corresponds to one physical
page of memory, such that the first entry in \texttt{pages} holds metadata
about the first page of physical memory, and so on.

Free pages are additionally stored in a linked list of \texttt{PageInfo}
structs called the \texttt{page\_free\_list}. Using a linked list lets the
kernel return a free page in constant time.

We wrote the following functions to manage physical pages:
\begin{itemize}
\item \texttt{page\_alloc} is used to allocate a page of physical memory
\item \texttt{page\_free} is used to put a page on the free list
\item \texttt{page\_decref} and \texttt{page\_incref} are used to manage
reference counts of pages
\end{itemize}
With this infrastructure in place, we were ready to set up the page table so
that we could use virtual memory.

% TODO: maybe describe our memory corruption checks?


\subsection{Page table theory}
To explain how our operating system implements virtual memory, it is necessary
to introduce some theory about the page table.

The page table is a two-level table whose main purpose is to let the processor
translate a virtual address to a physical address.  Conceptually, the x86 page
table is a 1024-ary tree of height 2. The physical address of the page table
-- the root of the tree -- can be found in the \gls{cr3} register.

The first level of the page table is called the Page Directory. It contains
1024 Page Directory Entries (\gls{pde}s). Each \gls{pde} points to a
second-level node in the page table.
Each second-level table holds 1024 Page Table
Entries (\gls{pte}s). It is possible to map a virtual address to a \gls{pte}.
The \gls{pte} then holds the physical address to which the virtual address
should map. It also holds some status bits.

The status bits of \gls{pte}s and \gls{pde}s determine such features as
whether the page is writable, and whether it is accessible to user-mode code.
There is also a "present" bit, which determines whether the page is virtual
address maps to a physical page at all, or if accesses should cause a page
fault instead.

To translate from a virtual to a physical address, it is necessary to walk the
page table. For sake of illustration, assume that the processor is instructed
to access a virtual address $v = 0x11223344$. The processor first looks in the
\gls{cr3} register to find the root of the page table. It then uses the
higher-order 10 bits of $v$ as an index into the page table. In this case, we
have:
$$ v = 0x11223344 = 0b10001001000100011001101000100 $$
So the 10 higher-order bits are:
$$ 0b1000100100 = 548 $$
At index 548 into the page table, the processor finds a \gls{pde}. It extracts
a physical address from the \gls{pde} to find a page directory. It then uses
the next 10 high-order bits of $v$ as an index into the page directory. We
have:
$$ 0b0100011001 = 281 $$
So the processor looks at index 281 into the page directory and finds a
\gls{pte}. The \gls{pte} holds some status bits, which the processor can use
to check whether the page is present, and whether access permissions are
appropriate. If not, a page fault is generated. If all goes well, the
\gls{pte} holds the physical address of the page that should be accessed. The
lower-order 12 bits of the virtual address are used to index into the physical
page, and the processor is finally done with address translation.

This is a costly process, and in practice the job is done by specialized
hardware called a Memory Management Unit (\gls{mmu}). Additionally, a cache
called the Translation Lookaside Buffer (\gls{tlb}) holds the results of
recent translations, to avoid having to walk the page table too often.


\subsection{Page table management}
\label{sec:pagetables}
We wrote the following functions to manage the page table:
\begin{itemize}
\item \texttt{pgdir\_walk} is the main workhorse of the subsystem, since it is
called by most of the other functions. It has the same function as the MMU; it
is given a page table and a virtual address, and it walks over the table to
find the corresponding \gls{pte}, allocating new levels of the table if needed,
using \texttt{page\_alloc} from the previous section.
\item \texttt{page\_insert} is used to insert a physical page into a page
table at a given virtual address. In other words, it finds a \gls{pte} for a virtual
address and stores the physical address there.
\item \texttt{page\_lookup} finds the physical address of a page, given a
virtual address.
\item \texttt{page\_remove} invalidates a \gls{pte} in a page table.
\end{itemize}
The kernel uses these functions to set up the page table. This involves
allocating pages of physical memory and inserting them into appropriate places
in the page table. 

It is up to us where the different things should go. Figure \ref{memlayout}
contains a diagram which gives a simplified overview of the address space. For
a more complete version, see the file \texttt{inc/memlayout.h}.
\begin{figure}[h!]
    \centering
\begin{Verbatim}[fontsize=\small]
Virtual memory map:                                  Permissions
                                                     kernel/user
   4 Gig -------->  +------------------------------+
                    :              .               :
                    :              .               :
                    |------------------------------| RW/--
                    |                              | RW/--
                    |      Kernel code, data       | RW/--
                    |                              | RW/--
   KERNBASE, ---->  +------------------------------+ 0xf0000000      
   KSTACKTOP        |     CPU0's Kernel Stack      | RW/--  KSTKSIZE 
                    +------------------------------+                 
                    |     CPU1's Kernel Stack      | RW/--  KSTKSIZE 
                    +------------------------------+                 
                    :              .               :                 
                    :              .               :                 
                    +------------------------------+ 0xef800000
                    |  Cur. Page Table (User R-)   | R-/R-  PTSIZE
   UVPT      ---->  +------------------------------+ 0xef400000
                    |          RO PAGES            | R-/R-  PTSIZE
   UPAGES    ---->  +------------------------------+ 0xef000000
                    :              .               :                 
                    :              .               :                 
   USTACKTOP  --->  +------------------------------+ 0xeebfe000
                    |      Normal User Stack       | RW/RW  PGSIZE
                    :              .               :                 
                    :              .               :                 
                    +------------------------------+
                    :              .               :
                    :              .               :
                    +------------------------------+
                    |     Program code, data       |
   UTEXT -------->  +------------------------------+ 0x00800000
                    :              .               :                 
                    :              .               :                 
                    +------------------------------+ 0x00000000
\end{Verbatim}
    \caption{The virtual address space of the kernel}
    \label{memlayout}
\end{figure}
The diagram shows that the kernel code resides starting at virtual address
0xf0000000. The kernel stacks, used by processors when running kernel-mode
code, reside just below, between 0xefc00000 and 0xf0000000. Further down in
the address space, between 0xef400000 and 0xef800000, we have the User Virtual
Page Table (UVPT) area, which gives user-mode programs read-only access to the
page table, enabling certain exokernel-style programs to work. The stack of
the user-mode program starts at 0xeebfe000 and grows toward lower addresses.
The user-mode program code and data resides at 0x00800000.

After allocating a page table and updating it to reflect all these
conventions, we update the \gls{cr3} register to atomically update the page
table. Finally we set a bit in \gls{cr0} to enable paging.

Our system now has virtual memory. This means that user-mode programs cannot
access kernel memory, since we have not set the user bit in the PTEs for this
memory. Furthermore, user-mode programs cannot interfere with each other; they
will each be given their own separate page table, where none of the physical
addresses overlap by default.


% TODO: make sure we describe that the user bit isn't set in kernel pages, and
% that's what gives us security.


%%%%% LAB 3 %%%%%
\section{Lab 3}

The goal of this lab was to get a user-mode program running as a process in
its own virtual address space. To accomplish this, we needed to write
functions to manage processes and another \gls{elf} file loader. If the process
triggers an exception, such as a division by zero, our kernel additionally
needs to handle this. Finally we implemented a system call mechanism to e.g.
let programs perform input and output.

\subsection{Process management}
Each process has some associated information. This includes its state
(running, runnable, killed, etc.), its process ID, its parent process ID, its
page directory, and so on. All this information is stored in a struct
\texttt{Env}. The process subsystem is similar to the page subsystem; an
array, \texttt{envs}, holds the \texttt{Env} struct of each process on the
system, and free environments are stored in a linked list called
\texttt{env\_free\_list}. We implemented functions for creating, initializing,
and destroying a process.

\subsection{ELF loading}
The kernel also needs a way to load a program into an address space. Programs
are represented as \gls{elf} files. This let us reuse our \gls{elf} loading code from our
boot loader.

To load a program, the kernel first allocates a fresh page table. The kernel
then walks over each section in the \gls{elf} file, figures out the virtual address
at which the section should go from the \gls{elf} file, allocates corresponding
physical pages, inserts them into the page table, and copies the code or data
from the program into the physical pages.

Note that we have not yet introduced a file system, so it is not immediately
clear where the kernel can find the programs which it should load. To solve
this problem we embed each user-mode program into the kernel as a blob of
binary data. In a later section we describe our implementation of a proper
file system for holding programs and data.

\subsection{Context switching}
To actually run a process, we need to perform a context switch. The
\texttt{Env} struct includes the metadata for a process, including its
registers. To perform a context switch, we first restore the saved
general-purpose registers by using the \texttt{popal} instruction. We also
load the page table of the process into \gls{cr3}.

We then issue the \texttt{iret} instruction, which restores the saved
\gls{eip}, \gls{esp}, \gls{eflags}, and \gls{cs} registers. This transfers
execution to user-mode code. The lower two bits of the \gls{cs} register
determine the Current Privilege Level (\gls{cpl}) of the processor. This was
previously 0, since the kernel runs in ring 0. By setting this to 3 during the
\texttt{iret}, the processor switches to user-mode operation, i.e., ring 3.

At this point we were able to run user-mode code in its own address space.
Unfortunately the code had no way to give control back to the kernel, so it
simply ran forever, or at least until it triggered an exception or interrupt,
which caused the whole system to crash.


\subsection{Theory of exceptions and interrupts}
While executing an instruction, the processor may trigger an exception. This
could e.g. happen due to a division by zero, or due to an illegal memory
access. When an exception occurs, the processor performs a context switch to
enter kernel mode. Then it runs some handler code specific to the exception.

The processor may also occasionally trigger an interrupt. This often happens
for asynchronous reasons; examples could be receiving a new network packet or
key press. Interrupts can also happen for synchronous reasons; for example, an
interrupt is raised as a result of executing the \texttt{int 0x80} instruction
which is commonly used for system calls. The behavior is the same for
exceptions; a context switch occurs to run a handler in kernel mode.

We now describe how the processor knows which code to run when an interrupt or
exception is raised. Exceptions and interrupts have numbers. Exceptions are
numbered from 0 to 31, and interrupts are numbered from 32 to 255.
% TODO: give some examples from http://wiki.osdev.org/Exceptions in a table.

These numbers specify an index into a table called the Interrupt Descriptor
Table (\gls{idt}). The physical address of the \gls{idt} is stored in the IDT
Register (\gls{idtr}), which can be read and set using the \texttt{lidt} and
\texttt{sidt} instructions, respectively.

The \gls{idt} table entry describes the address at which the handler for the
given interrupt resides.

As noted, an exception or interrupt triggers a context switch. Therefore the
processor needs to know on which stack it should save the registers of the
faulting process. To find this \gls{esp} value, the processor reads the Task
Register (\gls{tr}) which is an index into the \gls{gdt}. The \gls{gdt} entry
contains the address of a data structure called the Task State Segment
(\gls{tss}). The processor reads the new \gls{esp} value from the \gls{tss}.

To sum up: when an exception or interrupt occurs, the corresponding number is
looked up in the \gls{idt} to find the new \gls{eip} value. The \gls{tr},
\gls{gdt} and \gls{tss} are used to find the new \gls{esp} value. The
processor uses this information to store the registers of the faulting process
onto the new stack and execute the relevant handler code.


\subsection{Handling exceptions and interrupts}
We wrote a common function, \texttt{trap}, which is called whenever an
exception or interrupt occurs. For each exception and interrupt, we filled in
its \gls{idt} entry with a small stub which passes the number of the exception
or interrupt as the first argument in a call to \texttt{trap}.

The typical result of an exception is that the kernel terminates the running
process. However exceptions can also occur in kernel mode, in which case there
is a bug in the kernel. This results in a kernel panic, making the kernel
print the exception and hang.



\subsection{Handling system calls}
A process often needs to ask the kernel to perform some task for it. This
could e.g. involve printing text onto the console, reading a character from
the keyboard, or spawning a new child process. We needed to implement a
mechanism for triggering system calls in our kernel.

On the x86 architecture, the typical approach is to use the \texttt{int}
instruction, which triggers an interrupt when executed. On Linux, interrupt
0x80 is used for system calls, but we are free to use any interrupt, so we
arbitrarily chose number 0x30. This means that a program uses the \texttt{int
0x30} instruction to perform a system call. The interface was up to us, but we
kept it fairly standard; the \texttt{eax} register holds the system call
number, while arguments go in registers \texttt{ebx}, \texttt{ecx}, etc.

The \texttt{trap} function recognizes interrupt number 0x30 and calls the
\texttt{syscall} function, which uses a large \texttt{switch} to delegate each
system call to a specific handler.

We wrote system call handlers for input and output of a single character. At
this point we were able to run simple user-mode programs and have them
interact with the user. The programs could terminate themselves with a
specific system call, or by triggering any exception.


% TODO: also mention that our user-mode programs don't have much of a standard
% C library (yet)


%%%%% LAB 4 %%%%%
\section{Lab 4}
In the previous lab, we reached a point where our kernel could run a single
user-mode process. The current lab concerns itself with running multiple
processes concurrently, and having them interact. We implemented preemptive
multitasking, process forking, and inter-process communication. 
% TODO: this description misses a few of the following sections. We should
% update it to include more details.

Before we could do this, however, we needed to implement code for interacting
with a device called the LAPIC.


\subsection{Interacting with the LAPIC}
% source: 
% http://wiki.osdev.org/APIC
% https://en.wikipedia.org/wiki/Advanced_Programmable_Interrupt_Controller
A Programmable Interrupt Controlled (\gls{pic}) is a device which is
responsible for managing interrupts for the processor. For example, if
multiple interrupts are generated simultaneously, the \gls{pic} can prioritize
the interrupts and deliver them one at a time. When Intel updated their
\gls{pic} standard to include new features, the conforming device was called
an Advanced PIC (\gls{apic}). The \gls{apic} has a component called the Local
APIC (\gls{lapic}) which is local to each processor. Thus modern systems have
one \gls{lapic} per processor.

In section \ref{sec:moreprocs} we need to query the \gls{lapic} to activate
more processors than just the initial one. Furthermore, in section
\ref{sec:preempt} we need to ask the \gls{lapic} to generate periodic timer
interrupts which our scheduler will use to preempt the running process.
Therefore it is relevant to describe how our kernel communicates with the
\gls{lapic}.

The processor can communicate with its \gls{lapic} using Memory-Mapped I/O
(\gls{mmio}). This means that the \gls{lapic} is mapped into memory at a
specific, system-dependent physical address. Reading at certain offsets will
correspond to reading from certain registers in the \gls{lapic}, and likewise
for writing.

This means that to communicate with the \gls{lapic}, the kernel merely needs
to read and write to certain addresses. The difficult part is figuring out the
physical address where the \gls{lapic} resides.

There are multiple ways to find the \gls{lapic}. For now, our kernel uses the
method described in Intel's multi-processor specification. We will refer to
this as the \gls{mpconfig} method.
% TODO: insert reference to specification..

We leave out the details for brevity, but the method boils down to the
following. The kernel searches for a structure called the MP floating pointer
structure. It does so by looking in certain parts of physical memory for the
string "\_MP\_", validating a checksum on the memory to ensure that this was
not a false positive. This structure points to a table called the MP
configuration table, which contains the physical address of the \gls{lapic}. 
% TODO: mention that we were given mpconfig code



\subsection{Activating more processors}
\label{sec:moreprocs}
So far we have been running our kernel on an emulated machine with a single
processor. A system with $n$ processors can be emulated by passing the
\texttt{-smp n} option to QEMU.

When a system with multiple processors boots, the hardware dynamically selects
one of the processors to be the Bootstrap Processor (\gls{bp}). The other
processors are called Application Processors (\gls{ap}s). At first, only the
\gls{bp} runs. It is up to the \gls{bp} to start up the remaining \gls{ap}s
once the system is ready.

Up to this point, only the \gls{bp} had been running our kernel so far;
\gls{ap}s were never activated. We therefore needed to write code which starts
the \gls{ap}s. This is done by asking the \gls{lapic} of the \gls{bp} to send
an Inter-Processor Interrupt (\gls{ipi}) to each of the \gls{ap}s. This
\gls{ipi} causes the \gls{ap}s to start executing code. The address of the
code to run is sent as part of the \gls{ipi}.

% TODO: glossary-ize "real mode"
The \gls{ap}s start in 16-bit real mode, just as the \gls{bp} did. They
therefore need to switch to 32-bit real mode mode. After doing so, each
\gls{ap} calls into the scheduler to run a new process.

% source:
% https://stackoverflow.com/questions/14261612/which-core-initializes-first-when-a-system-boots
% which refers to an intel manual

\subsection{Ensuring mutual exclusion}
With multiple processors running concurrently, all the typical issues of
concurrency arose. Specifically, multiple processors could modify internal
kernel structures simultaneously, leading to race conditions.

We prevented this in the trivial but inefficient way: we make sure that at
most one processor is running kernel code at the same time. In other words,
our kernel is not truly concurrent. Still, user-mode processes \emph{can} run
truly concurrently.

The "big kernel lock" is a spinlock. It is essentially a global variable which
determines whether the kernel is locked. A processor repeatedly uses the
\texttt{lock} and \texttt{xchg} instructions to atomically exchange the global
variable with the value 1. If the global value was zero, the processor now
holds the lock and may enter the kernel. Otherwise it must retry.

We added calls to lock and unlock the kernel in the right places. At this
point our kernel was capable of running multiple user-mode processes
concurrently. However these processes ran forever, so we needed a scheduler
next.


\subsection{Process scheduling}
\label{sec:preempt}
The scheduler has as its main responsibility to pick a new process to run
whenever the current process is scheduled out. We opted for round-robin
scheduling.

That is, the scheduler keeps a circular queue of all processes. To find the
next process, the scheduler retrieves the next process from the queue until it
finds one that is runnable, at which point the scheduler is done.

The scheduler also needs a way to preempt each process when its time slice
runs out. To accomplish this, during kernel initialization the kernel asks the
\gls{lapic} to raise a timer interrupt periodically, waiting some fixed amount
of bus cycles between each raised interrupt. Our \texttt{trap} function then
recognizes this timer interrupt and reacts by asking the scheduler to schedule
in a new process.

Our scheduler has much room for improval. Round-robin scheduling is not as
performant as more advanced algorithms. Additionally, it does not allow
adjustment of process priorities. Scheduling being a linear-time operation is
potentially worrying. 

A final issue is that the time slice assigned to each process is always of the
same duration. It would be useful to assign shorter time slices and more
frequent schedulings to processes which need to feel responsive (such as
graphics-based processes introduced later) and longer time slices to processes
that perform heavy computations.


\subsection{A simple fork mechanism}
So far, every process was directly spawned by the kernel. However, processes
should also be able to spawn more processes. We therefore needed to implement
a mechanism to let a process fork. In this section we describe our initial,
simple implementation of fork, and in the following section we improve it by
introducing a copy-on-write mechanism.

Since our kernel is an exokernel, we prefer
to do as much work as possible outside of kernel land. We therefore wrote a
number of system calls which can be combined to implement a user-mode fork.
Specifically, we wrote the following system calls:
\begin{itemize}
\item	\texttt{sys\_exofork}: creates a non-runnable child process with an empty address space.
\item	\texttt{sys\_env\_set\_status}: can mark a process as runnable.
\item	\texttt{sys\_page\_alloc}: allocates an empty page in the address space of a process.
\item	\texttt{sys\_page\_map}: maps a page from the current process into a child process.
\item	\texttt{sys\_page\_unmap}: unmaps a page from the current process or a child process.
\end{itemize}
Besides these system calls, it is also necessary for a process to have access
to information about the layout of its own address space. This is already the
case; in section \ref{sec:pagetables} on page table management, we set up the
page table of a process such that part of the address space contains the page
table itself.

To fork, a parent process goes through the following steps:
\begin{itemize}
\item The parent process calls \texttt{sys\_exofork} to create a new child
process with an empty address space. The child process is not initially
runnable.
\item The parent walks over its page table, and for each mapped page, it does
the following:
\begin{itemize}
\item The parent uses \texttt{sys\_page\_map} to create a temporary page at a
temporary address.
\item The parent copies the contents of the current page into the temporary
page.
\item The parent uses \texttt{sys\_page\_map} to insert the temporary page
into the address space of the child process at the original address.
\item The parent uses \texttt{sys\_page\_unmap} to remove the temporary page
from its own address space.
\end{itemize}
\item The parent marks the child as runnable using
\texttt{sys\_env\_set\_status}. 
\end{itemize}
At this point the fork is complete, and since the child is marked as runnable,
the scheduler will eventually schedule it in.

Note that for security reasons, the system calls are coded to ensure that a
process cannot modify pages in other, non-child processes.


\subsection{Copy-on-write fork}
The simple version of fork described in the previous section is slow and
memory-inefficient, because it indiscriminately copies every page of the
parent into the child process. This involves a lot of copying, and it means
that if the parent used $n$ physical pages of memory, then after a fork, $2n$
physical pages will be used.

However the same physical page can transparently be mapped into both the
parent and child process, as long as it is never modified. In fact, \emph{all}
the pages in the child process can initially be shared with the parent. It is
only once a write happens that a page must be copied. We have implemented such
a copy-on-write fork mechanism almost entirely in user land, following the
exokernel design philosophy.

To perform as much work in user land as possible, we implemented a mechanism
which lets a process handle its own page faults. By default, the kernel will
terminate the running process if a page fault occurs. However we have
implemented a system call, \texttt{sys\_env\_set\_pgfault\_upcall}, which lets
a process set a handler function. If a handler is set, the kernel will handle
a page fault by modifying the saved \gls{eip} and \gls{esp} registers of the
process, pushing the old register values onto an exception stack, and
switching the process back in.

Now, when a process forks, \emph{all} pages are shared between the parent and
child process. However writable pages have their writable bit removed from
their page table entry (\gls{pte}). Instead we set another bit which marks the
page as copy-on-write.

If either the child or parent process attempts to write to said page, a page
fault will occur, since the page is not writable anymore. The kernel delegates
to the registered user-mode handler function. The handler then uses the same
method as in the simple fork implementation to map a new writable page, copy
the contents of the old page onto it, and replace the copy-on-write page with
the new writable page.


\subsection{Inter-process communication}
Since our kernel is an exokernel, many of the features implemented in later
labs will reside in user land. Two examples are a file system daemon, and a
daemon implementing a network stack. Other processes need a way to interact
with these daemons to make use of their services. We therefore have need of an
inter-process communication (\gls{ipc}) mechanism.

We therefore implemented two system calls, \texttt{sys\_ipc\_recv} and
\texttt{sys\_ipc\_try\_send}, which enable \gls{ipc}. When a process calls 
\texttt{sys\_ipc\_recv}, it will hang waiting for a process to send it data.
\texttt{sys\_ipc\_try\_send} will send data to a process in a non-blocking
fashion. By default, a 32-bit integer is sent between processes, but for
efficiency an extra argument to the system calls allows the sender to share a
full page of memory with the recieving process.


This marks the end of lab 4. Our kernel can now run user-mode processes in a
truly concurrent fashion. A scheduler manages the running processes,
preempting them when necessary. Processes can efficiently fork and communicate
via \gls{ipc}.


%%%%% LAB 5 %%%%%
\section{Lab 5}
The goal of this lab was to implement a custom file system and a shell.

\subsection{Custom file system}
So far, our kernel has embedded any user-mode programs inside itself as binary
blobs. This is highly undesirable; it requires recompilation of the full
kernel to modify user-mode programs, and it makes it impossible for programs
to store data persistently on disk. We therefore implemented a custom file
system, which we now describe.

A file system exists on a disk. A disk can be thought of as a large amount of
writable space. We can partition the disk's space into blocks, where each
block has a specific size. For our file system, the block size will be 4096
bytes.

One of these blocks is special; it's called the "superblock". The superblock
holds any metadata needed for the file system, such as the disk size, where to
find the root folder, etc.

Our file system is laid out as follows. The disk is partitioned into blocks as
mentioned. Block 0, the first one, is not used by our FS. (This way it can
hold the boot loader.) Block 1 is, by convention, the superblock. 

The next blocks, starting at block 2, hold a bitmap of free blocks; bits
correspond to blocks on disk, and each bit is 1 if, and only if, the
corresponding block is free. This way we can allocate and free blocks.

The remaining blocks are used to store the concrete files and folders.

We have decided to keep our file system as simple as possible for ease of
implementation. Thus there is no concept of permissions, symbolic and hard
links, timestamps, and so on.

% TODO: include graphics from https://pdos.csail.mit.edu/6.828/2016/labs/lab5/disk.png

A file is represented in C as a \texttt{struct file}, which is stored in a
block. Such a \texttt{file} struct contains metadata, such as the file name
and size. The struct also has 10 pointers to the blocks that hold
the raw file data. If the data cannot fit in 10 blocks, the \texttt{file}
struct has a pointer to a block which holds another 1024 pointers to data
blocks. Thus our file system has a maximum file size of $(10+1024)*4096
= 4235264$ bytes, i.e., around 4 MB.

% TODO: also this https://pdos.csail.mit.edu/6.828/2016/labs/lab5/file.png

Folders are represented as a \texttt{struct folder}, which is exactly
identical to a \texttt{struct file}, except that the $10+1024$ block
pointers no longer point to raw data, but to other blocks holding
\texttt{file} or \texttt{folder} structs. There is a type flag allowing us to
distinguish between files and folders.

We used a small script to create a raw disk image with an initialized file
system of the described format. The script let us add files, such as sample
programs, to the file system. We then attached this raw disk to QEMU.


\subsection{File system daemon}
In accord with exokernel design, we let all file system interaction go through
a single privileged process which we call the file system daemon.

The daemon interacts with the disk using \gls{pio}. However normal user land
processes cannot use the \texttt{inb} and \texttt{outb} instructions to
perform \gls{pio}, so our kernel needs to give the daemon I/O privileges. It
does so by setting a bit in the \gls{eflags} register for the daemon process.

The file system daemon spends its time looping, waiting for other processes to
contact it via \gls{ipc}. Processes can send requests to open, read, write,
and stat files. The \gls{ipc} details are hidden behind our implementation of
a C standard library, such that processes have the usual interface with
functions such as \texttt{open}, \texttt{read} and \texttt{write}.

We have further improved the efficiency of the file system daemon by
implemented a block caching system. When a block is first read, its contents
are stored in RAM, and subsequent reads do not need to interact with the disk
until the cache entry is invalidated through a write.



To summarize, when a process wants to open a file, the following happens. The
process calls the \texttt{open} library function. The library uses \gls{ipc}
to contact the file system daemon. The daemon uses the file system superblock
to find the root folder. It walks over the folder structure by following block
pointers, until it finds the block that holds the data for the correct file.
This location on disk is stored in a file descriptor, and the number of the
file descriptor is returned to the first process via \gls{ipc}. When a
subsequent \texttt{read} is performed, the daemon uses the pointers in the
\texttt{file} struct to find the raw data block and sends the requested data
back via \gls{ipc}.



\subsection{Shell}
We implemented a simplistic shell which lets us load and execute programs from
disk. The shell also features input redirection ("<"), output redirection
(">"), and pipes ("|"). It is now possible to write to the disk and have the
changes persist across reboots.

% TODO: we didn't describe the init process.

% TODO: show an example interaction.


%%%%% LAB 6 %%%%%
\section{Lab 6}
The goal of this lab was to connect our kernel to the internet by writing a
network card driver.

\subsection{The e1000 network card}
The operating system is connected to network via a network card. The network
card has as its main responsibility to send and receive packets by interacting
with the physical layer, such as a wire.

The network card emulated by QEMU is the Intel e1000. We therefore need to
write a driver for this card. The task of the driver is to discover the
network card, initialize and enable it, drain incoming packets from the card,
and deliver outgoing packets to the card.

We found all the details needed for the driver in the manual. It describes in
detail how the card works, how it should be initialized, and so on.
%TODO: insert reference to the manual and a proper name..

The e1000 card is connected through \gls{pci}. Our kernel uses \gls{pio} to
scan the PCI bus, iterating over all connected devices. Each device has
numbers representings its class, subclass, vendor ID and device ID. The kernel
uses these numbers to recognize the e1000 card if it is connected. The
\gls{pci} interface lets us determine a physical address at which the e1000
card is mapped. 

This completes the discovery phase; the kernel can now use \gls{mmio} to
communicate with the card. Concretely, this means that the various registers
of the network card can be found at specific offsets from the base address
found via \gls{pci}. In other words, the registers of the card can be read
from and written to by simply reading or writing at certain memory addresses.

The next step is initializing the card. We next describe the data structures
which the card uses, since these are what need to be initialized.

\subsection{Packet queues}
Overall, the e1000 network card uses two circular queues for holding packets.

One queue, the transmit queue, is for packets which are yet to be sent. When
the kernel wants to transmit a packet, it places it into this queue. The e1000
periodically drains this queue and puts the packets on the wire.

The other queue, the receive queue, is for packets which the e1000 has
received but which have not yet been claimed by the kernel. Thus when a new
packet arrives on the wire, the e1000 picks it up and deposits it into the
receive queue. The kernel then periodically drains this queue.

Each queue is implemented as an array of descriptor structs. A descriptor
contains a physical address and a size, and thus it describes a memory area
which can hold one packet. Each queue also has a tail index register and a
head index register which are used during insertion and removal of packets.

This means that when a packet arrives from the network, the e1000 card finds
the descriptor at the tail of the receive queue, copies the raw packet data
into the described memory area, and advances the tail register. The kernel
takes a packet out of the queue by copying the raw packet out of the
descriptor at the head of the queue, and then incrementing the head register.
Transmission of packets is analogous.

This raises the question of what should happen if one of the queues is full.
According to the manual, if the receive queue is full, the e1000 card will
simply drop further packets. However if the transmit queue is full, it is up
to the kernel what should be done. The kernel could potentially store packets
until the network card has drained the queue. However for simplicity we have
decided to simply drop packets if the transmission queue is full. We are free
to do this since the protocols on higher levels of the network stack handle
packet loss.

\subsection{The e1000 driver}
Before the card can be enabled, our driver must initialize these queues. That
is, it must allocate the physical pages required to hold the queues
themselves. It must also allocate pages for the raw packet contents, and fill
out the addresses and lengths into the descriptors in the queues. The head and
tail registers of each queue must also be reset. 

Once our driver has done this, it writes into a register to enable the card.
From this point it is possible to transmit and receive packets. Our driver
implements the code for taking packets from the receive queue and inserting
packets into the transmit queue. User-mode programs have access to this driver
functionality through two system calls, \texttt{sys\_transmit} and
\texttt{sys\_receive}.

% TODO: we should reorder these subsections; we should first describe the card
% in a general sense (here is how to discover it, here is how it works with
% queues, etc.).. and then, in a separate section, we should describe what our
% driver does.

% TODO: lesson: writing drivers is tricky. (debugging: QEMU source modification)


\subsection{The network daemon}
With the driver written, our kernel was capable of transmitting and receiving
packets. However most processes only know of the data they wish to send; they
are not capable of constructing packets.

We therefore needed a TCP/IP stack. Allegedly, writing such a network stack
from scratch is an immense task. We therefore used the open-source stack lwIP
("lightweight IP"). lwIP acts as a black box for our purposes, taking raw data
as input, and producing packets as output.

Rather than adding lwIP to the kernel, we embedded it in a network daemon. The
daemon is responsible for managing sockets, just as the file system daemon was
responsible for managing file descriptors. The network daemon takes in
\texttt{send} requests via IPC, produces packets, and hands these over to the
operating system via the \texttt{sys\_transmit} system call. Likewise, it
takes in \texttt{receive} requests and uses \texttt{sys\_receive}, parses the
resulting packets, and hands over the data to the other process.

The IPC communication is hidden away in the C standard library, such that
user-mode programs have access to the usual \texttt{connect}, \texttt{send}
and \texttt{receive} interface.

% TODO: we have not written about MACs and ip address assignment..


\subsection{Web server}
To test the new functionality, we wrote a simple web server which can serve
files from the file system. The web server thus makes use of both the file
system daemon and the network daemon. 

We configured QEMU to forward requests at port 80 to the emulated machine. At
this point we were able to point a browser at the machine and be served a
working web page.


%%%%% GRAPHICS LAB %%%%%
\section{Graphics Lab}
- graphics

%%%%% HARDWARE LAB %%%%%
\section{Hardware Lab}
- hardware



- conclusion

% TODO: we could include screenshots to make things more concrete
% TODO: include (a link to) the source code somehow

% TODO: question: is it a good idea to have "mistakes made" and "lessons
% learned"? If so, how do we do this without it sounding too informal?

% TODO: we should talk about future work.
% - security: ASLR, probably many other things (it's not even a multi-user
% system yet)
% - capabilities?
% - basic programs like an editor, browser, etc.
% - better windowing system (faster algorithms, better fonts, z-coordinate..)
% - could get a better scheduler
% - could have true kernel concurrency
% - better file system (timestamps, ...)


% TODO: add references (some links are in writeup txt files)


\begin{framed}
\makebox[\textwidth]{Lesson learned: foo} \\
% TODO:
Maybe we can use something like this to describe the various lessons we
learned...
\end{framed}



\begin{thebibliography}{1}

% use like this: \cite{sigar1, sigar2}

% \bibitem{proc}
% Procyon,
% \\\texttt{http://bitbucket.org/mstrobel/procyon}
% 
% \bibitem{em} 
% Exam Monitor website.
% \\\texttt{http://exammonitor.dk}
% 
% \bibitem{jnlp}
% Exam Monitor JNLP file.
% \\\texttt{http://login.exammonitor.dk/exam.jnlp}
% 
% \bibitem{sigar1}
% Hyperic SIGAR website.
% \\\texttt{https://support.hyperic.com/display/SIGAR/Home}
% 
% \bibitem{sigar2}
% libsigar at GitHub.
% \\\texttt{https://github.com/hyperic/sigar}
% 
% \bibitem{sigarapache}
% The license file of SIGAR.
% \\\texttt{https://github.com/hyperic/sigar/blob/master/LICENSE}
% 
% \bibitem{agents}
% The package java.lang.instrument.
% \\\texttt{https://docs.oracle.com/javase/7/docs/api/java/lang/instrument/package-summary.html}
% 
% \bibitem{javassist}
% Javassist website.
% \\\texttt{http://www.javassist.org}
% 
% \bibitem{openjdk}
% OpenJDK website.
% \\\texttt{http://openjdk.java.net/}
% 
% \bibitem{webportal}
% Exam Monitor web portal.
% \\\texttt{https://sdu.exammonitor.dk/myexams.php}
% 
% \bibitem{x86harmful} 
% Joanna Rutkowska. Intel x86 considered harmful. 2015.
%  
% \bibitem{sokintro} 
% Bhushan Jain, Mirza Basim Baig, Dongli Zhang, Donald E. Porter, and Radu Sion.
% SoK: Introspections on Trust and the Semantic Gap. 2014.
% 
% \bibitem{appshield} 
% Yueqiang Cheng, Xuhua Ding, Robert H. Deng. AppShield: Protecting Applications
% against Untrusted Operating System. 2013.
% 
% \bibitem{towardsappsec} 
% Dan R. K. Ports, Tal Garfinkel. Towards Application Security on Untrusted
% Operating Systems. 2008.
% 
% \bibitem{haven} 
% Andrew Baumann, Marcus Peinado, and Galen Hunt. Shielding Applications from an
% Untrusted Cloud with Haven. In \emph{Proceedings of the 11th USENIX Symposium
% on Operating Systems Design and Implementation.} 2014.
% 
% \bibitem{sokhiee} 
% Fengwei Zhang, Hongwei Zhang. SoK: A Study of Using Hardware-assisted Isolated
% Execution Environments for Security. In \emph{Hardware and Architectural
% Support for Security and Privacy (HASP '16)}. 2016.
% 
% \bibitem{sgxexplained} 
% Victor Costan and Srinivas Devadas. Intel SGX Explained. 2016.
% 
% \bibitem{malwareinenclaves} 
% Jeroen van Prooijen. The Design of Malware on Modern Hardware: Malware Inside
% Intel SGX Enclaves. 2016.
% 
% \bibitem{enclavesinpractice} 
% JP Aumasson, Luis Merino. SGX Secure Enclaves in Practice: Security and Crypto
% Review. Presented at \emph{Black Hat USA 2016}.
% 
% \bibitem{opensgx} 
% Prerit Jain, Soham Desai, Seongmin Kim, Ming-Wei Shih, JaeHyuk Lee,
% Changho Choi, Youjung Shin, Taesoo Kim, Brent Byunghoon Kang, Dongsu Han.
% OpenSGX: An Open Platform for SGX Research. 2016.
% 
% \bibitem{iago} 
% Stephen Checkoway, Hovav Shacham. Iago Attacks: Why the System Call
% API is a Bad Untrusted RPC Interface. 2013.

\end{thebibliography}

\printglossary[title=Abbreviations]



\end{document}

